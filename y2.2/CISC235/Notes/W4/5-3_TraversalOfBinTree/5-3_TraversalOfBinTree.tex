\documentclass[12pt]{book} 

\usepackage{amsmath}
\usepackage{graphicx}
\usepackage{import}

\setlength{\parindent}{0em}  % sets auto indent at new paragraph to none

\newcommand{\incfig}[1]{%
    \import{./figures/}{#1.pdf_tex}
}

\title{\coursetitle\linebreak\lecturename}
\author{\\Cain Susko\\ 
           \\ \\ \\
      Queen's University 
    \\School of Computing\\} 

%=-=-=-=-=-title-=-=-=-=-=%
\newcommand{\lecturename}{Traversal of a Binary Tree}
\newcommand{\coursetitle}{Data Structures}
%=-=-=-=-=-#####-=-=-=-=-=%

\begin{document}
\begin{titlepage}
        \maketitle
\end{titlepage}


\section*{Traversal}
Traversal means to visit every node in a tree, any time we want to traverse a tree we start from the root. There are 4ways
        to traverse a tree.

\subsection*{Preorder Traversal}
the algorithm for this operation is:
\begin{enumerate}
        \item process current node
        \item process the nodes in the left subtree
        \item process the nodes in the right subtree
\end{enumerate}

Note: process means something like the following:
\begin{itemize}
        \item print the current node
        \item search node to see if its value matches a target
        \item add node value to sum
        \item  etc\ldots
\end{itemize}

This algorithm is implemented recursively as one processes a node and then follows the path down to the leftmost node
        and then works its way back up processing the right subtrees of each node that was processed in the left step.
The path can be summarized as following the perimiter of the tree counterclockwise. A node is read when the path passes the left
        side of a node.

\subsection*{Inorder Traversal}
\begin{enumerate}
        \item process the nodes in the left subtree
        \item process the current node
        \item process the nodes in the right subtree
\end{enumerate}
if you draw a counterclockwise path around the circle, the node is read when the path passes underneath a node

\subsection*{Postorder Traversal}
This algorithm is very similar to the Preorder trvaersal.
\begin{enumerate}
        \item Process the nodes in the left subtree
        \item process the nodes in the right subtree
        \item process the current node
\end{enumerate}
using a similarly counterclockwise path, a node is only read when the path passes to the right of a node.

\subsection*{Level order Traversal}
in a level order traversal algorithm we visit each level's nodes from left to right before then visiting the nodes on the next level.
This algorithm is non-recursive.
The order in which the nodes are read can be determined by defining each row and reading top-down left-right.

\section*{Implementation}
One could use a Queue in order to traverse a tree like so
\begin{enumerate}
        \item use a temo variable and queue
        \item insert the root node into the queue
        \item while the queue is not empty
        \begin{enumerate}
                \item dequeue the top node and put it in temp
                \item process the node
                \item enqueue the nodes children to the queue if the are not None
        \end{enumerate}
\end{enumerate}
\end{document}


