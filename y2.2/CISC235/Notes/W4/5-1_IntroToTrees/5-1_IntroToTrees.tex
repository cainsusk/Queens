\documentclass[12pt]{book} 

\usepackage{amsmath}
\usepackage{graphicx}
\usepackage{import}

\setlength{\parindent}{0em}  % sets auto indent at new paragraph to none

\newcommand{\incfig}[1]{%
    \import{./figures/}{#1.pdf_tex}
}

\title{\coursetitle\linebreak\lecturename}
\author{\\Cain Susko\\ 
           \\ \\ \\
      Queen's University 
    \\School of Computing\\} 

%=-=-=-=-=-title-=-=-=-=-=%
\newcommand{\lecturename}{Introduction to Trees}
\newcommand{\coursetitle}{Data Structures}
%=-=-=-=-=-#####-=-=-=-=-=%

\begin{document}
\begin{titlepage}
        \maketitle
\end{titlepage}


\section*{Trees}
Trees are used as a data structure for many applications like
\begin{itemize}
        \item databases
        \item data compression
        \item etc\ldots
\end{itemize}

A tree is a special \textbf{linked} data structure that has many uses in Computer science.
Normally these can store hierarchical data and make that data easily searchable.
Some examples include parse trees, Binary trees, or Expression Trees (Like in cisc221 lesson 25).
\begin{figure}[h]
        \centering
        \incfig{tree1}
\end{figure}

\subsection*{properties}
a tree has some basic properties:
\begin{itemize}
        \item trees are made of nodes.
        \item the top or origin of the tree is called the root node.
        \item every node may have 0 or more child nodes.
        \item a node with 0 children is called a leaf node
        \item a tree with no nodes in an empty tree.
\end{itemize}

Note: a node can have \textbf{more} that 2 nodes.

\subsection*{edge and path}
a tree ca represent edges and paths.
An edge is a likage between 2 nodes or a node and a leaf. A path is a seires of edges one could take to get from the higher node to the
        lower node.

Note that a tree is onmidirectional--you can only go down.

\subsection*{height of node}
the height of a node is the number of \textit{edges} in the \textbf{longest} downward path from said node.
Any node can have a height. A leaf has a height of 0 (essentially no height).

Essentially, it is the longest path to a leaf from a node.

Furthermore, the height of a tree is the height of its root.

\subsection*{depth}
The depth of a node is the number of edges from said node to the root of the tree.

\subsection*{level}
the level of a node is the layer of the tree it is in, starting counting from one.

Simply, is is the depth of a node plus 1




\end{document}


