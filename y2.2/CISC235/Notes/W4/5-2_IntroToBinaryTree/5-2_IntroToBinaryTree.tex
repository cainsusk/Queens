\documentclass[12pt]{book} 

\usepackage{amsmath}
\usepackage{graphicx}
\usepackage{import}

\setlength{\parindent}{0em}  % sets auto indent at new paragraph to none

\newcommand{\incfig}[1]{%
    \import{./figures/}{#1.pdf_tex}
}

\title{\coursetitle\linebreak\lecturename}
\author{\\Cain Susko\\ 
           \\ \\ \\
      Queen's University 
    \\School of Computing\\} 

%=-=-=-=-=-title-=-=-=-=-=%
\newcommand{\lecturename}{Introduction to Binary Trees}
\newcommand{\coursetitle}{Data Structures}
%=-=-=-=-=-#####-=-=-=-=-=%

\begin{document}
\begin{titlepage}
        \maketitle
\end{titlepage}


\section*{Valid Trees}
A valid tree is one that is defined as a tree where there is a single unique path to each node.

\section*{Binary Tree}
A binary tree is a tree where each node either has 2 to 0 children.
A binary tree can represent an arithmetic expression can be represented as a binary tree whose leaves are associated with 
        variables or constants and whose internal nodes are associated with one of the operators in the expression.

\subsection*{Binary Subtrees}
a sub tree is a section of a Binary tree that is also a valid tree, where the root of the subtree is a node in the original tree.

\subsection*{Complete \& Full Binary Trees}
A \textbf{full} binary tree is a tree where each node is either a leaf or possesses exactly two child nodes.

A \textbf{complete} binary tree is one where all \textit{levels} are completely filled \textbf{except} the last level which has all 
        keys as left as possible.

A Binary tree can be both full or complete.

\section*{Operations}
the following are common operations that one might perform on a binary tree
\begin{itemize}
        \item traverse all items
        \item search for item
        \item adding a new item
        \item deleting item or the entire tree (destruction)
        \item removing or adding a section of the tree
\end{itemize}

\section*{example}
Below is an example implementation of a Binary tree class in Python
\begin{verbatim}
class BTNode:
        def_init_(self, data):
                self.left = None
                self.right = None
                self.value = data

def main():
        BTNode temp

        pRoot = new BTNode(5)

        temp = new BTNode(7)
        pRoot.left = temp

        temp = new BTNode(-3)
        pRoot,right = temp
\end{verbatim}

\end{document}

