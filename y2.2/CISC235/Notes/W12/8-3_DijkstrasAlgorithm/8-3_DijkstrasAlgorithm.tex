\documentclass[12pt]{book} 

\usepackage{amsmath}
\usepackage{graphicx}
\usepackage{import}
\usepackage{amsfonts}
\usepackage{booktabs}

\setlength{\parindent}{0em}  % sets auto indent at new paragraph to none

\newcommand{\incfig}[1]{%
        \import{./figures/}{#1.pdf_tex}
}

\newcommand{\incimg}[2]{%
       \begin{figure}[h]
               \centering
               \includegraphics[scale = #2]{./figures/#1}
       \end{figure}
}

\title{\coursetitle\linebreak\lecturename}
\author{\\Cain Susko\\ 
           \\ \\ \\
      Queen's University 
    \\School of Computing\\} 

%=-=-=-=-=-title-=-=-=-=-=%
\newcommand{\lecturename}{Djikstra Algorithm}
\newcommand{\coursetitle}{Data Structures}
%=-=-=-=-=-#####-=-=-=-=-=%

\begin{document}
\begin{titlepage}
        \maketitle
\end{titlepage}


\section*{Finding the Shortest Path}
it is useful to know the shortest path in an a graph is
useful as it allows us to move as efficiently as possible
in a graph. A graph will normally be wieghted if you
are asked to find the shortest path, the objective is to 
find the path from $a$ to $b$ with the smallest sum of weights.

One can also be taked to find the shortest path between any 2 verticies
in a graph.

\section*{Djikstra's Algorithm}
This algorithm determines the shortest path in $G$ from vertex $s$ to
\textbf{all other} verticies in the graph.
\begin{align*}
        &\text{Input} &G, s\\
        &\text{Output} &dist
\end{align*}

where $dist$ is an array containing the optimal distances from $s$
to every other vertex in $G$.

\subsection*{Data Structure}
Djikstra uses 3 different data structures to carry out its operation.
\begin{enumerate}
        \item A $dist$ array (min-heap) that tracks the \textbf{current
                best known cost} to get from $s$ to every other vertex in
                the graph. For each vertex $i$ in $dist$:
        \begin{enumerate}
                \item set the distance from $s$ to be equal 
                        to 0
                \item set the distance from all other verticies to $\infty$
        \end{enumerate}
        \item An array called $done$ that holds boolean values denoting
                if a vertex has been fully prosessed. For each vertex $i$,
                set $done[i]$ to false.
        \item An array called $parent$ that holds the parent for each
                vertex.
\end{enumerate}
\pagebreak

\subsection*{Algorithm}
The Djikstra Algorithm is as follows:
\begin{verbatim}
while !done {
        u = the closest unprocessed vertex to s

        for v in notDoneVertx{
                if edgeExists.(u,v){
                        if dist[v] > dist[u] + weight(u,v){
                                dist[v] = dist[u] + weight(u,v)

                                parent[v] = u
                        }
                }
        }
}

\end{verbatim}

\end{document}

