\documentclass[12pt]{book} 

\usepackage{amsmath}
\usepackage{graphicx}
\usepackage{import}
\usepackage{amsfonts}
\usepackage{booktabs}

\setlength{\parindent}{0em}  % sets auto indent at new paragraph to none

\newcommand{\incfig}[1]{%
        \import{./figures/}{#1.pdf_tex}
}

\newcommand{\incimg}[2]{%
       \begin{figure}[h]
               \centering
               \includegraphics[scale = #2]{./figures/#1}
       \end{figure}
}

\title{\coursetitle\linebreak\lecturename}
\author{\\Cain Susko\\ 
           \\ \\ \\
      Queen's University 
    \\School of Computing\\} 

%=-=-=-=-=-title-=-=-=-=-=%
\newcommand{\lecturename}{Djikstra Algorithm}
\newcommand{\coursetitle}{Data Structures}
%=-=-=-=-=-#####-=-=-=-=-=%

\begin{document}
\begin{titlepage}
        \maketitle
\end{titlepage}


\section*{Finding the Shortest Path}
it is useful to know the shortest path in an a graph is
useful as it allows us to move as efficiently as possible
in a graph. A graph will normally be wieghted if you
are asked to find the shortest path, the objective is to 
find the path from $a$ to $b$ with the smallest sum of weights.

One can also be taked to find the shortest path between any 2 verticies
in a graph.

\section*{Djikstra's Algorithm}
This algorithm determines the shortest path in $G$ from vertex $s$ to
\textbf{all other} verticies in the graph.
\begin{align*}
        &\text{Input} &G, s\\
        &\text{Output} &dist
\end{align*}

where $dist$ is an array containing the optimal distances from $s$
to every other vertex in $G$.

\subsection*{Data Structure}
Djikstra uses 3 different data structures to carry out its operation.
\begin{enumerate}
        \item A $dist$ array (min-heap) that tracks the \textbf{current
                best known cost} to get from $s$ to every other vertex in
                the graph. For each vertex $i$ in $dist$:
        \begin{enumerate}
                \item set the distance from $s$ to be equal 
                        to 0
                \item set the distance from all other verticies to $\infty$
        \end{enumerate}
        \item An array called $done$ that holds boolean values denoting
                if a vertex has been fully prosessed. For each vertex $i$,
                set $done[i]$ to false.
        \item An array called $parent$ that holds the parent for each
                vertex.
\end{enumerate}
\pagebreak

\subsection*{Algorithm}
The Djikstra Algorithm is as follows:
\begin{verbatim}
while !done {
        u = the closest unprocessed vertex to s

        for v in notDoneVertx{
                if edgeExists.(u,v){
                        if dist[v] > dist[u] + weight(u,v){
                                dist[v] = dist[u] + weight(u,v)

                                parent[v] = u
                        }
                }
        }
}
\end{verbatim}

\subsection*{Relaxation Step}
Relaxing the edge ($u,v$) means to test whether we can improve the shortest path to $v$ by going through $u$

\section*{Time Complexity}
The Time complexity analysis is as follows:
\incimg{time}{0.3}

\section*{Explaination}
{\small
Let the node at which we are starting be called the initial node. 
Let the distance of node Y be the distance from the initial node to Y. 
Dijkstra's algorithm will initially start with infinite distances 
and will try to improve them step by step.
\begin{enumerate}
        \item Mark all nodes unvisited. Create a set of all the unvisited nodes called the unvisited set.
        \item Assign to every node a tentative distance value: set it to zero for our initial node and to infinity for all other nodes. The tentative distance of a node v is the length of the shortest path discovered so far between the node v and the starting node. Since initially no path is known to any other vertex than the source itself (which is a path of length zero), all other tentative distances are initially set to infinity. Set the initial node as current.
        \item For the current node, consider all of its unvisited neighbors and calculate their tentative distances through the current node. Compare the newly calculated tentative distance to the current assigned value and assign the smaller one. For example, if the current node A is marked with a distance of 6, and the edge connecting it with a neighbor B has length 2, then the distance to B through A will be 6 + 2 = 8. If B was previously marked with a distance greater than 8 then change it to 8. Otherwise, the current value will be kept.
        \item When we are done considering all of the unvisited neighbors of the current node, mark the current node as visited and remove it from the unvisited set. A visited node will never be checked again.
        \item If the destination node has been marked visited (when planning a route between two specific nodes) or if the smallest tentative distance among the nodes in the unvisited set is infinity (when planning a complete traversal; occurs when there is no connection between the initial node and remaining unvisited nodes), then stop. The algorithm has finished.
        \item Otherwise, select the unvisited node that is marked with the smallest tentative distance, set it as the new current node, and go back to step 3.
\end{enumerate}
When planning a route, it is actually not necessary to wait until the destination node is "visited" as above: the algorithm can stop once the destination node has the smallest tentative distance among all "unvisited" nodes (and thus could be selected as the next "current").
\footnote{Wikipedia}
}
\end{document}

