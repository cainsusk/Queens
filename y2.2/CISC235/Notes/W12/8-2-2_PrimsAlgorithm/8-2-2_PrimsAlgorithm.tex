\documentclass[12pt]{book} 

\usepackage{amsmath}
\usepackage{graphicx}
\usepackage{import}
\usepackage{amsfonts}
\usepackage{booktabs}

\setlength{\parindent}{0em}  % sets auto indent at new paragraph to none

\newcommand{\incfig}[1]{%
        \import{./figures/}{#1.pdf_tex}
}

\newcommand{\incimg}[2]{%
       \begin{figure}[h]
               \centering
               \includegraphics[scale = #2]{./figures/#1}
       \end{figure}
}

\title{\coursetitle\linebreak\lecturename}
\author{\\Cain Susko\\ 
           \\ \\ \\
      Queen's University 
    \\School of Computing\\} 

%=-=-=-=-=-title-=-=-=-=-=%
\newcommand{\lecturename}{Minimum Spanning tree}
\newcommand{\coursetitle}{Data Structures}
%=-=-=-=-=-#####-=-=-=-=-=%

\begin{document}
\begin{titlepage}
        \maketitle
\end{titlepage}


\section*{Time Complexity}
The time complexity for an MST with $n$ elements is:
\begin{align*}
        &\text{Create} &O(1)\\
        &\text{Find} &O(\log n)\text{ or }O(1)\\
        &\text{Union} &O(\log n)\text{ or }O(1)\\
\end{align*}
the last 2 are $O(1)$ if the targets are already found. (ie. already found root $a$ and $b$ for Union)

\section*{Prim's Algorithm}
This is an algorithm to grow a MST in successive steps:
\begin{enumerate}
        \item given the set $S$ of the $n$ edges we are choosing
        \item randomly choose a vertex $v \in T$ and the rest of the vertices are in the set $Q$
        \item while the size of $S$ is less than $n-1$:
                \begin{enumerate}
                        \item let $e$ be the edge with the lowest weight which connects $v$ to any vertex in $Q$ (ie. $e=(x,y)$ where $x$ is in 
                                $T$ and $y$ is in $Q$)
                        \item add $e$ to $S$
                        \item add $y$ to $T$
                        \item remove $y$ from $Q$
                \end{enumerate}
        \item return $S$
\end{enumerate}
Note: this  is a `greedy'/`local' algorithm for taking a 
graph and converting it into a MST. 

\paragraph*{Finding Lightest Edge}
a key part of Prim's algorithm is finding the lightest edge in a Tree. To quickly do this, we are given a list/dictionary $K$ which stores
a priority-vertex pair. $K[v]$ stores the `shortest distance' between $v$ and the current tree. The algorithm is as follows:
\begin{enumerate}
        \item After adding a new node to the tree $T$ we update the weights of all nodes adjacent to it
        \item $K[v] = \infty$ if $v$ is not adjacent to any vertices in $T$
\end{enumerate}

Consider the following example of Prim's algorithm where we are given a graph, and it is gradually converted into
MST:
\incimg{example}{0.42}

This is just an example of 2 iterations of the algorithm, going from $b$ to $c$, and choosing $i$ as the next vertex with the 
lightest edge.

The final result for this graph would be:
\incimg{result}{0.55}
\pagebreak

\section*{Prim's v. Kruskal's}
For either algorithm that is used to make a MST, there are pro's and cons.
\incimg{PvK}{0.5}

It is also good to note that both algorithms will always give
the same solutions with the same length, even if the order in which
the vertices are chosen is different. Occasionally, the algorithms will
choose different edges but this is ony when either option is the same
length.

\end{document}

