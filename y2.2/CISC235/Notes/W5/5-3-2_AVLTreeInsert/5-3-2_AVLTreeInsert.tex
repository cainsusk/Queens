\documentclass[12pt]{book} 

\usepackage{amsmath}
\usepackage{graphicx}
\usepackage{import}
\usepackage{amsfonts}
\usepackage{booktabs}

\setlength{\parindent}{0em}  % sets auto indent at new paragraph to none

\newcommand{\incfig}[1]{%
        \import{./figures/}{#1.pdf_tex}
}

\newcommand{\incimg}[2]{%
       \begin{figure}[h]
               \centering
               \includegraphics[scale = #2]{./figures/#1}
       \end{figure}
}

\title{\coursetitle\linebreak\lecturename}
\author{\\Cain Susko\\ 
           \\ \\ \\
      Queen's University 
    \\School of Computing\\} 

%=-=-=-=-=-title-=-=-=-=-=%
\newcommand{\lecturename}{Inserting Nodes Into an AVL Tree}
\newcommand{\coursetitle}{Data Structures}
%=-=-=-=-=-#####-=-=-=-=-=%

\begin{document}
\begin{titlepage}
        \maketitle
\end{titlepage}


{\small
\section*{Operations}
For all AVL operations it is assumed that the tree was balanced \textit{before} the operation began.

\section*{Insert}
The insertion process begins the same as with a normal BST, but adding a new node to the AVL tree, it 
may unbalance the tree by 1. This would cause the AVL tree to become invalid. We rectify this by using an operation
called Rotation.

\paragraph{Rotation}
If a node has becore out of balance in a given direction, we \textit{rotate} it in the \textbf{opposite} direction.
This maintains the inorder ordering of keys. The height of each touched node must be updated as well.

\paragraph{Unbalanced Cases}
Consider the lowest node $k$ that has now become unbalanced. The new `offending' node could be in one of the four
following \textit{grandchild} subtrees relative to $k$:
 \begin{enumerate}
        \item[\textbf{Left / Right}]
                There was an insert into a left child's left subtree or insert into right childs right subtree. 
                All that is required is to rotate once away from the direction of the offending node.
        \item[\textbf{L-R / R-L}] There was an insert in the left subtree of a right child or a right subtree of a 
                left child. There first must be a rotation on the offending new node 
                (opposide of its direction), and then a rotation on the offending, existing child node 
                (again, opposide to its direction).
\end{enumerate}

\subsection*{Algorithm}
thus, the insertion algorithm for AVL trees is the following:
\begin{enumerate}
        \item Find a Location within a given tree for the new value that satisfies the existing BST requirements.
        \item Insert the node.
        \item Search from the insert up the tree, looking for imbalances.
        \item if an imbalance is found:
        \begin{enumerate}
                \item if it is \textbf{Left / Right}, perform a single rotation.
                \item if it is \textbf{L-R / R-L}, perform a double rotation
        \end{enumerate}
\end{enumerate}
 }

\end{document}

