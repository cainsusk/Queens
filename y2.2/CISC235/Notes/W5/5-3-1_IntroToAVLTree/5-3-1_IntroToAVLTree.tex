\documentclass[12pt]{book} 

\usepackage{amsmath}
\usepackage{graphicx}
\usepackage{import}
\usepackage{amsfonts}
\usepackage{booktabs}

\setlength{\parindent}{0em}  % sets auto indent at new paragraph to none

\newcommand{\incfig}[1]{%
        \import{./figures/}{#1.pdf_tex}
}

\newcommand{\incimg}[2]{%
       \begin{figure}[h]
               \centering
               \includegraphics[scale = #2]{./figures/#1}
       \end{figure}
}

\title{\coursetitle\linebreak\lecturename}
\author{\\Cain Susko\\ 
           \\ \\ \\
      Queen's University 
    \\School of Computing\\} 

%=-=-=-=-=-title-=-=-=-=-=%
\newcommand{\lecturename}{Into to AVL Trees}
\newcommand{\coursetitle}{Data Structures}
%=-=-=-=-=-#####-=-=-=-=-=%

\begin{document}
\begin{titlepage}
        \maketitle
\end{titlepage}


\section*{Balanced BST}
A slight problem with Binary Search Trees is that if they are unbalanced (ie. The 
left subtree of the root-node is larger than the right) then the efficiency 
of a BST can be diminished. 
This is because if a tree is unbalanced there is a high lekelyhood that
the data is organized in an ordered list rather than the intended tree structure.
The solution to this problem is the AVL tree.

\section*{AVL Tree}
An AVL tree is a Binary Search Tree that uses modified indert and delete operations
to stay Balanced as its elements change. It does this by maintaining a Balance 
factor in each node of 0, 1, or -1.

An AVL Tree is more suitable to search through often (rather than the Red Black tree,
which will be covered later, which is more suited for often inserts and deletes).

\paragraph{Balance Factor}
the balance factor is, for a tree node $T$, the height of the node  $T$'s right
subtree minus the height of the left subtree.
 \[
BF(T) = H(T.R) - H(T.L)
.\] 
This can be interpereted as we only allow at most 1 node difference between the 
two subtrees. Note, the height is the max number of \textit{edges} between a given
node and a leaf. Additionally, the height of an empty tree is -1.

We focus on height so much because the Big O, or complexity for a search, is increased
by how high the tree is, as is  cannot have more than 2 nodes off of one node and 
has a max width for each height,

\paragraph{Height}
The Height of an AVL tree with $N$ nodes is  $O(\log N)$. To Prove this, observe the 
following:

Let us bound $n(h)$ which is the minimum number of nodes an AVL tree of height  $h$
can have. The base cases are:
\[
n(0) = 1 \;\; n(1) = 2 \;\; n(2) = 4
.\] 

From this basis we can say that for $n(h)>2$, an AVL Tree of height $h$ has, at minimum,
the root node with one AVL subtree of height  $h-1$ and another AVL subtree of height
 $h-2$. Thus $n(h)$ for AVL subtrees can be generalized as for AVL subtrees can be generalized as:
 \[
 n(h) = 1 + n(h-1) + n(h-2) \;\;\;\;\; n(h) > 2^i*n(h-2i)
 .\] 

 However, we can even further simplify the inequality on the right by removing the
 variable $i$ by claiming that  $i = \lceil \frac{h}{2}-1 \rceil $. If $h$ is even
 then,  $h-2i = h-(h-2) = 2$. And if  $h$ is odd then  $h-2i = h - \left(2 \lceil \frac{h}{2} \rceil -2\right) \implies h - ((h+1)-2 = 1 $. Thus, we can generalize
 the inequality to the following:
 \[
 n(h) \geq 2^{(\frac{h}{2})-1}
 .\] 

 Additionally, a tree of height $h$ has a search complexity of  $O(\log N)$ where
  $N$ is the number of nodes in said tree.

The Height of any given node is the maximum between its left and right subtree
plus 1. This means that there is a relationship bewteen different nodes' height and
the height could be calculated recursively. 

However, within an AVL tree it is common
to maintain a height for each node as a field within the node which is altered 
whenever the nodes' height is altered as the height is needed often to measure
and maintain the balance factor of the tree (and thus, each of its nodes).

\end{document}

