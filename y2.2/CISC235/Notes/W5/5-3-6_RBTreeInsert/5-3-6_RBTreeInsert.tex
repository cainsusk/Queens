\documentclass[12pt]{book} 

\usepackage{amsmath}
\usepackage{graphicx}
\usepackage{import}
\usepackage{amsfonts}
\usepackage{booktabs}

\setlength{\parindent}{0em}  % sets auto indent at new paragraph to none

\newcommand{\incfig}[1]{%
        \import{./figures/}{#1.pdf_tex}
}

\newcommand{\incimg}[2]{%
       \begin{figure}[h]
               \centering
               \includegraphics[scale = #2]{./figures/#1}
       \end{figure}
}

\title{\coursetitle\linebreak\lecturename}
\author{\\Cain Susko\\ 
           \\ \\ \\
      Queen's University 
    \\School of Computing\\} 

%=-=-=-=-=-title-=-=-=-=-=%
\newcommand{\lecturename}{Red Black Tree Insert and Delete}
\newcommand{\coursetitle}{Data Structures}
%=-=-=-=-=-#####-=-=-=-=-=%

\begin{document}
\begin{titlepage}
        \maketitle
\end{titlepage}


\section*{Intro to RBTree Insert and Delete}
the insert and delete operations modify the tree such that we must fix the following in
order for it to remain a RBTree.
\begin{enumerate}
        \item colour changes
        \item restructuring the links of the tree via. rotation
\end{enumerate}

\section*{Rotations}
rotations are used to maintain the ordering of the keys. A rotation can be performed in $O(n)$

\section*{Insertion}
When doing an insertion, the only property that is violated is property 4 (see lesson 5-3-5).
This is because nodes are only added as a leaf of the tree (as is the normal).
There are 3 cases for this violation given $x$, the target node:
\begin{enumerate}
        \item if the Aunt or Uncle node $y$ of $x$ is red then:\\
                recolour $y$, the parent of $y$, and the parent of $x$
        \item if the Aunt or Uncle node is black, and $x$ and it's parent are on the same
          side, then:\\
                right-rotate $y$ and recolour the parent of $x$ and $y$
                \incimg{yBlack}{0.5}
        \pagebreak

        \item if the Aunt or Uncle node is black, and $x$ and it's parent are on different sides,
        then:\\
               left rotate $x$ 
               \incimg{case3}{0.5}
\end{enumerate}

The time complexity of all 3 of these cases are:
        \[O(\log n)\]
to see each of these cases in the context of eachother in a real example, observe the following:
\incimg{bigEx}{0.5}

\paragraph{Algorith for Insertion}
from all this information, we can generalize the process for insertion as the following algorithm:
\begin{verbatim}
if current and parent are both red:
        if grandparent's other child (aunt or uncle) is red:
                colour grandpa red
        else if the current and parent are both on the same side:
                do a single rotation and recolour
        else:
                do a double rotation and recolour
\end{verbatim}

An implementation can also be seen in the following figure
\incimg{insEx}{0.5}
\incimg{LLfix}{0.3}
\incimg{LRfix}{0.3}
\incimg{AllCases}{0.6}

\end{document}

