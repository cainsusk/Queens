\documentclass[12pt]{book} 

\usepackage{amsmath}
\usepackage{graphicx}
\usepackage{import}
\usepackage{amsfonts}
\usepackage{booktabs}

\setlength{\parindent}{0em}  % sets auto indent at new paragraph to none

\newcommand{\incfig}[1]{%
        \import{./figures/}{#1.pdf_tex}
}

\newcommand{\incimg}[2]{%
       \begin{figure}[h]
               \centering
               \includegraphics[scale = #2]{./figures/#1}
       \end{figure}
}

\title{\coursetitle\linebreak\lecturename}
\author{\\Cain Susko\\ 
           \\ \\ \\
      Queen's University 
    \\School of Computing\\} 

%=-=-=-=-=-title-=-=-=-=-=%
\newcommand{\lecturename}{Data Structures}
\newcommand{\coursetitle}{AVL Tree Implementation}
%=-=-=-=-=-#####-=-=-=-=-=%

\begin{document}
\begin{titlepage}
        \maketitle
\end{titlepage}


\section*{Update Height}
it is important to keep track of and update the height of an AVL tree,
as mentioned in the lesson 5-3-2. the height is mainly changed during insertion and deletion.
an implementation of this process could look like so. Note that this code is within a ALVTree
class.
{\scriptsize
\begin{verbatim}
# step 1, perform normal BST insert
def insert(self, root, key):
        if not root:
                return TreeNode(key)
        elif key < root.val:
                root.left = self.insert(root.left, key)
        else:
                root.right = self.insert(root.right, key)

# step 2, Update the height of the ancestor node

        root.height = 1 + max(self.Height(root.left), self.Height(root.right))

# step 3, get the balance factor of the node

        balance = self.getBalance(root)

# step 4, if node is unbalanced then try out the 4 cases

# CASE 1 - LeftLeft
if balance < -1 and key < root.left.val:
        return self.rightRotate(root)

# CASE 2 - RightRight
if balance > 1 and key > root.right.val:
        return self.leftRotate(root)

# CASE 3 - LeftRight
if balance < -1 and key > root.left.val:
        root.left = self.lefttRotate(root.left)
        return self.rightRotate(root)

# CASE 4 - RightLeft
if balance > 1 and key < root.right.val:
        root.right = self.rightRotate(root.right)
        return self.leftRotate(root)

\end{verbatim} 
}
\pagebreak

the implementation for the rotate functions mentioned above are the following:

\incimg{rotate}{0.5}


\end{document}

