\documentclass[12pt]{book} 

\usepackage{amsmath}
\usepackage{graphicx}
\usepackage{import}
\usepackage{amsfonts}
\usepackage{booktabs}

\setlength{\parindent}{0em}  % sets auto indent at new paragraph to none

\newcommand{\incfig}[1]{%
        \import{./figures/}{#1.pdf_tex}
}

\newcommand{\incimg}[2]{%
       \begin{figure}[h]
               \centering
               \includegraphics[scale = #2]{./figures/#1}
       \end{figure}
}

\title{\coursetitle\linebreak\lecturename}
\author{\\Cain Susko\\ 
           \\ \\ \\
      Queen's University 
    \\School of Computing\\} 

%=-=-=-=-=-title-=-=-=-=-=%
\newcommand{\lecturename}{Red And Black Trees}
\newcommand{\coursetitle}{Data Structures}
%=-=-=-=-=-#####-=-=-=-=-=%

\begin{document}
\begin{titlepage}
        \maketitle
\end{titlepage}


\section*{Definition of RBT}
An Red Black Tree is another type of binary search tree that is self balancing. Each node in the
tree is either coloured $red$ or $black$. In order for a tree to be a RBT it must satisfy the
following conditions
\begin{enumerate}
        \item Every node is red or black
        \item the root is black
        \item every leaf (NIL) is black
        \item if a node is red, then both of it's children are black. thus no 2 consecitive nodes
                are red.
        \item for each node $x$, all simple paths from the node to it's descendant leaves
                contain the same ammount of black nodes.
\end{enumerate}

\incimg{RBTex}{0.5}

An RBTree also has the following properties:
\begin{itemize}
        \item The path from any node to the farthest leaf is no more than \textit{twice} as long
                as the path from the node to the nearest leaf
        \item the sortest path for node $n$ down to a leaf is equal to $k'\geq b \geq \frac{k}{2}$
                where $k'$ is the length of the shortest path and $b$ is the number of 
                black nodes in all paths (as this number must be the same for all paths).
        \item a red black tree with nodes has a height of:
                \[h \geq 2\log_2 (n+1)\]
        \item the number of leaves in an RBTree is $n+1$
\end{itemize}

\end{document}

