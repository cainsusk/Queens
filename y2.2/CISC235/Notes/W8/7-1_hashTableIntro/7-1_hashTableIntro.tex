\documentclass[12pt]{book} 

\usepackage{amsmath}
\usepackage{graphicx}
\usepackage{import}
\usepackage{amsfonts}
\usepackage{booktabs}

\setlength{\parindent}{0em}  % sets auto indent at new paragraph to none

\newcommand{\incfig}[1]{%
        \import{./figures/}{#1.pdf_tex}
}

\newcommand{\incimg}[2]{%
       \begin{figure}[h]
               \centering
               \includegraphics[scale = #2]{./figures/#1}
       \end{figure}
}

\title{\coursetitle\linebreak\lecturename}
\author{\\Cain Susko\\ 
           \\ \\ \\
      Queen's University 
    \\School of Computing\\} 

%=-=-=-=-=-title-=-=-=-=-=%
\newcommand{\lecturename}{Intro to Hash Tables}
\newcommand{\coursetitle}{Data Structures}
%=-=-=-=-=-#####-=-=-=-=-=%

\begin{document}
\begin{titlepage}
        \maketitle
\end{titlepage}


\section*{Hash Table}
a hash table is a data structure that implements a map ADT (associative array). a hash table has the properties of:
\begin{itemize}
        \item a structure that can map keys to values
        \item a structure for fast lookup (and insertion)
\end{itemize}

and a map ADT consists of:
\begin{itemize}
        \item a set of unique keys
        \item a set of values where each key is associated with one value or set of values. for example:
                \[\texttt{\{('cain', 12), ('matthew', 11)\}}\]
\end{itemize}
Search engines use hashmaps to find webpages containing words in your query.
An example of a hash table used in a search engine is a \textbf{doc-word }$<$key-value$>$ pairs:
\incimg{keyval}{0.6}
\pagebreak

\section*{Map Operations}
so to go over the hash table operations we must first go over the map operations:
\begin{itemize}
        \item get(key k)\\
                returns null if key is not in map
        \item put(key k, val v)\\
                if the key is already associated with a value, replace that value with val
        \item remove(key k)\\
                if key is not in the map, do nothing
        \item size()
        \item isEmpty()
\end{itemize}

the main idea of a hash table is that we can use some mathematical function that takes a input (normally number) and converts it into a unique slot
number between 0 and 99,999 in an array.

\paragraph{Hash Function}
the main way of keeping the output of the hash within the array bounds is by using the modulus function. once we have that implemented we can
then create a operation for mutating the input such that each input gives a unique output for the given application

\end{document}

