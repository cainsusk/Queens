\documentclass[12pt]{book} 

\usepackage{amsmath}
\usepackage{graphicx}
\usepackage{import}
\usepackage{amsfonts}
\usepackage{booktabs}

\setlength{\parindent}{0em}  % sets auto indent at new paragraph to none

\newcommand{\incfig}[1]{%
        \import{./figures/}{#1.pdf_tex}
}

\newcommand{\incimg}[2]{%
       \begin{figure}[h]
               \centering
               \includegraphics[scale = #2]{./figures/#1}
       \end{figure}
}

\title{\coursetitle\linebreak\lecturename}
\author{\\Cain Susko\\ 
           \\ \\ \\
      Queen's University 
    \\School of Computing\\} 

%=-=-=-=-=-title-=-=-=-=-=%
\newcommand{\lecturename}{Heap Sort}
\newcommand{\coursetitle}{Data Structures}
%=-=-=-=-=-#####-=-=-=-=-=%

\begin{document}
\begin{titlepage}
        \maketitle
\end{titlepage}


\section*{The Heapsort}
given an array of $N$ numbers that we want to sort:
\begin{enumerate}
        \item Insert all $N$ numbers into a new maxheap
        \item wile there are nunbers left in the heap:
                \begin{enumerate}
                        \item remove the largest value from the heap.
                        \item place it in the last open slot in the new maxheap
                \end{enumerate}
\end{enumerate}
However, this is a niave approach, as we should convert the input into a maxheap:
\incimg{maxheapSort}{0.5}

for step 1 the algorithm is:
\begin{enumerate}
        \item for (currentNOde = lastNode till rootNode):\\
                focus on the subtree rooted at currentNode.\\
                think of this subtree as a maxheap\\
                keep shifting the top value down until your subtree becomes a valid maxheap
\end{enumerate}
once we finish heapisfying from our \textit{root node}, our entire array will hold a valid maxheap!
however, because the array represents a complete binary tree, we dan ignore all nodes before $N/2 -1$ as all of the nodes after are leafs, so
$lastNode = N/2 - 1$
\pagebreak

Now, for step 2, we will use the reheapify algorithm as follows;
\incimg{reheap}{0.5}

\end{document}

