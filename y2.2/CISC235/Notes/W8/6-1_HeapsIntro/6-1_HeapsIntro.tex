\documentclass[12pt]{book} 

\usepackage{amsmath}
\usepackage{graphicx}
\usepackage{import}
\usepackage{amsfonts}
\usepackage{booktabs}

\setlength{\parindent}{0em}  % sets auto indent at new paragraph to none

\newcommand{\incfig}[1]{%
        \import{./figures/}{#1.pdf_tex}
}

\newcommand{\incimg}[2]{%
       \begin{figure}[h]
               \centering
               \includegraphics[scale = #2]{./figures/#1}
       \end{figure}
}

\title{\coursetitle\linebreak\lecturename}
\author{\\Cain Susko\\ 
           \\ \\ \\
      Queen's University 
    \\School of Computing\\} 

%=-=-=-=-=-title-=-=-=-=-=%
\newcommand{\lecturename}{Heaps}
\newcommand{\coursetitle}{Data Structures}
%=-=-=-=-=-#####-=-=-=-=-=%

\begin{document}
\begin{titlepage}
        \maketitle
\end{titlepage}


\section*{Heaps}
heaps are a type of data structure that is based on the binary tree and a priority queue. there are 2
types of a heap: Maxheap and Minheap

\paragraph{Maxheap}
a max heap can quickly insert values and quickly get the largest value inthe heap.

within a maxheap, data is contained such that a nodes children are all less then the value in the node. it is also a complete binary tree.

\paragraph{Minheap}
a min heap can quickly insert into heap and get the smallest value from the heap,

the structure is the same as maxheap but with the min at the root with all children are lagrer.

\subsection*{Operations}
the main operation with a heap is \textbf{extraction}. The algorithm for extraction is as follows:
\incimg{extract}{0.5}

\subsection*{Implementation}
we can use arrays in order to implement a heap.

we can map the complete binray tree of the heap using level order traverse and entering the values by row. the memory for each row increses eponentially
(accept for possibly the final row).

with this implementation, we can always find the largest value at \texttt{HEAP[0]} and the bottom-right-most mode at \texttt{HEAP[-1]}.
finally, we can also always find the bottom-left-most empty spot (for adding a new value) at \texttt{HEAP[len(HEAP)]}.

but how do we maintain the tree structure within the array?specifically, can we find the children of a node using an algorithm like so:
\[\texttt{leftChild(parent) = 2 * parent + 1} \] 
\[\texttt{rightChild(parent) = 2 * parent + 2}\]

So, the code implementation may use the following logic:
\incimg{extractCode}{0.5}
\pagebreak

Additionally, the implementation for inserting a node to the heap would be:
\incimg{addCode}{0.5}

\subsection*{Complexity}
since our tree is a \textit{complete} binary tree, if it has $N$ entries, it is guarunteed to be exactly $log_2(N)$ Levels deep.

extraction, takes, for the max/min: $O(1)$ and for any other value the complexity is $O(\log n)$

the complexity for searching for an element in a heap, however, is $O(n)$
\incimg{complexity}{0.5}











\end{document}

