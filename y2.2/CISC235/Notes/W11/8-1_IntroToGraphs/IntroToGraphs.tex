\documentclass[12pt]{book} 

\usepackage{amsmath}
\usepackage{graphicx}
\usepackage{import}
\usepackage{amsfonts}
\usepackage{booktabs}

\setlength{\parindent}{0em}  % sets auto indent at new paragraph to none

\newcommand{\incfig}[1]{%
        \import{./figures/}{#1.pdf_tex}
}

\newcommand{\incimg}[2]{%
       \begin{figure}[h]
               \centering
               \includegraphics[scale = #2]{./figures/#1}
       \end{figure}
}

\title{\coursetitle\linebreak\lecturename}
\author{\\Cain Susko\\ 
           \\ \\ \\
      Queen's University 
    \\School of Computing\\} 

%=-=-=-=-=-title-=-=-=-=-=%
\newcommand{\lecturename}{Introduction to Graphs}
\newcommand{\coursetitle}{Data Structures}
%=-=-=-=-=-#####-=-=-=-=-=%

\begin{document}
\begin{titlepage}
        \maketitle
\end{titlepage}


\section*{Graphs}
Grpahs are made up of nodes which can be anythinf from data, to a person, to 
another network). The nodes are connected via edges which define with 
nodes are related to which.
\incimg{graph}{0.7}

\section*{Graph Reoresentation}
A graph can be represented as either an adjacency matrix or edge list:
\incimg{Amat}{0.5}

Because of the flexibility of the edge list, which is a linked list of arrays, 
it is a better option to implement and represent a graph in code. However, if 
you know your graph will be quite full, this flexibility isn't used and the speed
from pre-allocating an array (adjacency matrix) will be more helpful.

Graphs can be both dense or sparse, an example of a sparse graph is a subway map
like so:
\incimg{subway}{0.5}
\pagebreak

\section*{Graph as ADT}
A conceptual Graph would have the following operations:
\begin{itemize}
        \item Test if graph is empty
        \item Get number of nodes in graph
        \item Get number of edges in graph
        \item See whether edge exists between 2 nodes
        \item Insert node into graph
        \item Insert edge between nodes in graph
        \item Remove node from graph and any edges to said node
        \item Remove edge between two nodes in graph
\end{itemize}

\incimg{code}{0.5}

\section*{Traversal}
like with a tree (which is almost a kind of graph) there are may
way to traverse a graph:

\paragraph{Breadth first Search}
This way explores the graph in \textbf{growing concentric circles}, exploring a
node's first edge, then their second, etc...Iterating over Graph

\paragraph{Depth first Search}
A depth first search keeps moving forward until it hits a dead end or a 
previously-visited node. It then backtracks and tries another path.

\subsection*{Breadth First Implementation}
\incimg{bftCode}{0.5}


\end{document}

