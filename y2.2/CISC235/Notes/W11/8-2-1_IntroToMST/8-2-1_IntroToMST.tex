\documentclass[12pt]{book} 

\usepackage{amsmath}
\usepackage{graphicx}
\usepackage{import}
\usepackage{amsfonts}
\usepackage{booktabs}

\setlength{\parindent}{0em}  % sets auto indent at new paragraph to none

\newcommand{\incfig}[1]{%
        \import{./figures/}{#1.pdf_tex}
}

\newcommand{\incimg}[2]{%
       \begin{figure}[h]
               \centering
               \includegraphics[scale = #2]{./figures/#1}
       \end{figure}
}

\title{\coursetitle\linebreak\lecturename}
\author{\\Cain Susko\\ 
           \\ \\ \\
      Queen's University 
    \\School of Computing\\} 

%=-=-=-=-=-title-=-=-=-=-=%
\newcommand{\lecturename}{Intro To Minimum Spanning Trees}
\newcommand{\coursetitle}{Data Structures}
%=-=-=-=-=-#####-=-=-=-=-=%

\begin{document}
\begin{titlepage}
        \maketitle
\end{titlepage}


\section*{Spanning Trees}
A spanning Tree is, given the graph $G$, a sub-graph of $G$: $T$ which
contains all vertices within $G$. It is known that every connected graph 
(there is a path from any node to any other node) has at least one
spanning tree.

\paragraph{Tree Generation}
There are a few ways to convert a (connected) graph into a spanning
tree. These include Depth First and Breadth First search which are both
ways of creating a tree by traversing it.

\section*{Minimum Spanning Tree}
The idea of a minimum spanning tree is, given a graph with weighted edges,
what is the lowest `cost' (sum of weights) for a spanning tree. For example,
given the graph:
\incimg{graph}{1}

If we were to create a few spanning trees we could then find the 
minimum one:
\incimg{MST}{0.7}

And thus, tree 2 is the MST for the given graph. Note: there is another tree
not shown which has a weight of 74.

\subsection*{Definition}
A minimum Spanning Tree from an undirected, connected, weighted graph is
a spanning tree of minimum weight. There may be more than one MST for
a given graph.

\paragraph{Applications}
The concept of the MST is intriguing for the computer scientist as it
allows for the association of data with the least amount of memory 
required for maintaining the relationships. The applications can include
minimizing communication networks to optimizing transit networks.

\section*{Kruskal's Algorithm}
A basic algorithm for growing an MST from a graph is as follows: given
a graph $G$,
\begin{verbatim}
def Kruskal(G) {
        S=[]
        while |S| < n-1{
                Let (u,v) be the minimum weighted edge in G
                        such that S.append((u,v)) has no cycles

                S.append((u,v))
        }
        return S
}
\end{verbatim}

Which is to say, one adds edges in increasing weight, skipping those who's
addition would create a cycle, thus making it \textit{not} a tree.

\paragraph{Checking For Cycles}
by intuition, if one is to connect 2 vertices which are already in the same 
graph, this creates a cycle. In order to avoid a vertex being added which is
already in the tree we must give each vertex a label and keep track of such
labels using the Union-Find data structure.

\subsection*{Union-Find}
This is a type of data structure which uses the tree ADT. The root of
the tree is a representative item of the entire tree. Each node below
the root points back to its parent; the root points to itself.

There are 3 operations for Union-Find:
\begin{itemize}
        \item \texttt{Create-Set(u)} - create a set containing the single
                element \texttt{u}
        \item \texttt{Find-Set(u)} - find the set containing the element
                \texttt{u}
        \item \texttt{Union(u,v)} - merge the sets respectively 
                containing \texttt{u} and \texttt{v} into a common
                set \texttt{a}
\end{itemize}

Thus, Using this data-structure, Kruskal's algorithm is the following:
\begin{verbatim}
^^ Sorted edgeList By Weight ^^

A = []

for each u in V{
        create-set(u)
}

for (u,v) in edgeList{

        if (find-set(u) != find-set(v)){
                A.append((u,v))

                Union(u,v)
        }
}
\end{verbatim}

\paragraph{Implementation}
For each of the tree operations, the general implementation is
the following:
\begin{itemize}
        \item create: \texttt{u.parent = u}
        \item find: 
                \begin{verbatim}
                temp = u.parent
                while temp.parent != temp{
                        temp = temp.parent
                }
                return temp
                \end{verbatim}
        \item union: 
                \begin{verbatim}
                a = find-set(x)
                b = find-set(y)

                if (a.height <= b.height){
                        if (a.height == b.height){
                                b.height++
                        }
                        a.parent = b
                } else {
                        b.parent = a
                }

                \end{verbatim}
\end{itemize}

The next note goes over a different algorithm for MST generation.


\end{document}

