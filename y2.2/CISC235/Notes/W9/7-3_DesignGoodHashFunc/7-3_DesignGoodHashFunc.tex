\documentclass[12pt]{book} 

\usepackage{amsmath}
\usepackage{graphicx}
\usepackage{import}
\usepackage{amsfonts}
\usepackage{booktabs}

\setlength{\parindent}{0em}  % sets auto indent at new paragraph to none

\newcommand{\incfig}[1]{%
        \import{./figures/}{#1.pdf_tex}
}

\newcommand{\incimg}[2]{%
       \begin{figure}[h]
               \centering
               \includegraphics[scale = #2]{./figures/#1}
       \end{figure}
}

\title{\coursetitle\linebreak\lecturename}
\author{\\Cain Susko\\ 
           \\ \\ \\
      Queen's University 
    \\School of Computing\\} 

%=-=-=-=-=-title-=-=-=-=-=%
\newcommand{\lecturename}{How To Design A Good HAsh Function}
\newcommand{\coursetitle}{Data Structures}
%=-=-=-=-=-#####-=-=-=-=-=%

\begin{document}
\begin{titlepage}
        \maketitle
\end{titlepage}


\section*{Algorithms}
The djb2 algorithm was first reproted by Dan Berstien many years ago.
\incimg{djb2}{0.5}

\paragraph{Python Hash}
In python, a hhash function could look like so:
\incimg{hashpy}{0.5}

\section*{Designing Your Own Hash Function}
When Creating a  Hash Function a large part of the process is figureing out what data to use from the input it hash the data. This is because is
is common for the Hash Table to contain a besspoke dataa  type for storing a specific class of data. The ocnvention is to design your algorithm,
\textit{analyze it}, and \textbf{iterate it}
\begin{itemize}
        \item[\textbf{Tip 1}] Make Use of All Data, as this help avoid collisions
        \item[\textbf{Tip 2}] Try to Spread Out Values so that they are more Evenly Distributed.
\end{itemize}

In practice, is is very hard to derive the hash values independent of any patterns (which is what were trying to do with the quadratic probing and
double hashing).

There are a few methods we can use in order to try and achive this even distrubution.

\paragraph{Division Method}
The idea is to map a key $k$ to one of the$m$ slots by taking the remainder of $k$ divided by $m$.
\[h(k) = k\mod m\]
Although this method is fast, prime values of $m$ cannot be used.

\paragraph{Multiplication Method}
The algorithm for this method is generally as follows:
\begin{enumerate}
        \item Multiply each key $k$ by a constant $A$, where $0 < A < 1$
        \item extract the fractional part of $A$, $kA$
        \item multiply fractional part $kA$ by $m$
        \item take the floor of the above result:
                \[h(k) = \lfloor m(kA - \lfloor kA \rfloor n ( k A \mod 1)\rfloor \]
\end{enumerate}

\pagebreak
\section*{Examples}
for Division
\incimg{ex1}{0.4}

and Multiplication
\incimg{ex2}{0.4}



\end{document}

