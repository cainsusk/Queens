\documentclass[12pt]{book} 

\usepackage{amsmath}

\newcommand{\week}{Week 3}
\newcommand{\class}{Urban Planning}

\begin{document}
\date{}
\setlength{\parindent}{0em}  % sets auto indent at new paragrahp to none

\title{\class\\\week}

\author{\\ \\ Cain Susko\\\today \\ \\ \\ \\ \\
Queen's University \\Faculty of Arts \& Science} 
 

\maketitle
\pagebreak


\section*{3 A Critical Introduction to Urban Geography}

this week will cover the following:
\begin{enumerate} 
        \item the rationalle for critical approaches to urban geograph
        \item how a critical approach is different from others
        \item understanding gentrification through cirtical analysis
\end{enumerate}


\subsection*{3.1 Why a Critical Approach to the City?}

\paragraph{}
in the 50s and 60s geography was a quantatative discipline using positivist approaches.

before then, geography was a science of descrition and environment, where each city is unique. 

This shifted to be a science of statistics and 'laws'.
This was known as the 'quantatative revolution', which sought to isolate facts about urban areas and derive
universal laws from them.

\paragraph{}
In the 70's there was pushback from this quantatative approach due to social injustice in the city.

The focus now was how wider social relations affect the space in which people live.
A famous book on the matter is David Harvey's " \textit{Social Justice in the City}.

This critical change occured because of social turmoil caused by discrimination \& racism. 
The turmoil was expressed through rebellions against the police in sevral US cities. 
Many were killed and entire street blocks were razed in the The Newark \& Detriot Rebellions.

\paragraph{}
according to survivors of the rebellion, the revolt had a profound impact on peoples views on human rights.
Very soon after the riots, the first black mayor of Newark was elected.

Between 1960-1971 nealy 1000 similar--albeit smaller--urban uprisings occured. 
This is mainly thought to be caused by both the civil unrest of the south and the white flight and segregation in the north.

the converse result of these revolutions is that there was a right-wing counter revolution that imposed a war on crime and drugs.
This 'war' is well known to have dispropotionate effect on people of colour.

\paragraph{}
in summary, the critical approach to urban geography focuses on the problems \textit{of} the city, rather than with.
The critical approach emphasizes the wider economic context that allows or prohibits the owning of property
        and how the private property market reflects, enforces, and exacerbates existing economic inequality.


\subsection*{3.2 Cities for Whom?}
a critical geographic approach questions the power, distrubution, and justice of a city in order to understand urban environments
the components of a critical approach are:
\begin{enumerate}
        \item Proposing new concepts that shed light on wider social issues
        \item socially relevant and politically engaging research
        \item taking seriously the experiences of ordinary urban residents and\\
                marginaized social groups
\end{enumerate}

item 2. specifically pretains to this section's title, 'cities for whom?

within this approach there are many different frameworks.
these frameworks allow us to see different aspects of urban life.
they are 
\begin{enumerate}
        \item structuralist
        \item post-modern
        \item post-colonial
\end{enumerate}

These are heuristic devices; which is to say a simplified way of discussing and modelling the world.
This is similar to an abstraction.

\subsubsection{Structuralism}
this structure examines how urban processes are shaped by capitalist economic system, examining 
        the relation between cities and their economies (as Harvey says).
Often draws upon Marxis theory which focuses on the 'mode of production'

\subsubsection{Post-Modernism}
This structure emphasizes human difference and studying the point of view of many people within the city.
These can be seen as Queer, Black, etc. geography. an example is how muslims felt in urban spaces after 911.

\subsubsection{Post-Colonialism}
Questions the eurocentric understandings of urban development as well as the ongoing effect colonialism has on urban life.
A consideration within this structure is to 'decentre' the refrence points for urban geographic research.

\paragraph{}
these structures can be combined to further analyze an urban space with regards to social wellbeing.
In summary, these approaches are 'urban' because they question the hegemonic rhetoric currently being taught and 
        evaluate the status quo. this is urban because the city is constantly changing and has no status quo.

in other words, the urban is shaped by, and shapes wider social, economic, political, and environmental processes.


\subsection*{3.3 Gentrification: a Critical Concept}
Gentrification is the process by which urban neighborhoods become the focus of reinvestment and (re)settlement by 
        the middle class.
Socially, it is frequently represented by the displacement of existing residents  and higher taxes.

Displacement is the continued exclusion of residents from direct eviction, or indirect expulsion.

\paragraph{}
The conventional backdrop for gentrification is the inner city but it can also occur in sleepy towns.
An example of gentrification is that of the changing Parkdale, Toronto.
This place used to be lived in by immagrants and lower income people. But with huge new investments from large
        firms the cost of living and way of life in Parkdale has been completely changed.

Another example is The Albert residence in Detroit where senior residents were evicted--nearly homeless--in order
        to build an upscale apartment. there is money being made at the cost of these seniors' lives.

\paragraph{}
While some parts of Gentrification are good, like increased investment, better services, and mixing of classes.
In reality, the increased investment in services only increases property value, which pushes out existing residents.
Thus there is no mixing, or there is for a short time.

\subsection*{3.4 Understanding Gentrification through Critical Geography}
we can interperet the causes of gentrification using the frameworks above in many ways.
\begin{enumerate}
        \item natural processes
        \item economic (Structural)
        \item desires and gentrifiers (post-modern)
        \item racist and colonial logic (post-colonial)
\end{enumerate}

\subsubsection{Natural}
This argues factors like street networks and other itrinzic parts of the city.
Thus stating that gentrification is a natural part of the city.

\subsubsection{Economic}
argues that the economic climate effects gentrification.
for example, a rent gap can allow for gentrification in a neighborhood by making properties more profitable by selling them
        or redeveloping. the more gentrification, the greater the rent gap.

\subsubsection{Desires and Gentrifiers}
argues that there are those who are 'gentrifiers' who have desires to fufill. 
an example of this is young professionals desiring a more exiting and urban lifestyle, gentrifying the down town. 

\subsubsection{Racism and Colonial Logic}
Argues that raceism and discrimination is the largest factor in gentrification, as who is in power directly 
        effects those either displaced or displacing.
\textit{not in my neighborhood} is an interesting documentary covering the gentrification of 3 cities around the world.

\paragraph{}
in summary, gentrification is a global issue felt by millions around the world and can be interperted in countless ways by using
        the approach and frameworks defined above. 
Generally speaking though, these combine into a problem of how can a neighborhood be desireable, but not gentrifiable.
the advent of AirBnB and Real Estate Investment Trusts (REITs) has further added considerations to the question of gentrification
        as they have both widened the conditions for a viable and desirable property--that is, for investment.


\end{document}
