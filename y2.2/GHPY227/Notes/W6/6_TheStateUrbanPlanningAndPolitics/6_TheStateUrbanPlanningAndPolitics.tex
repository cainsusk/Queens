\documentclass[12pt]{book} 

\usepackage{amsmath}
\usepackage{graphicx}
\usepackage{import}
\usepackage{amsfonts}
\usepackage{booktabs}

\setlength{\parindent}{0em}  % sets auto indent at new paragraph to none

\newcommand{\incfig}[1]{%
        \import{./figures/}{#1.pdf_tex}
}

\newcommand{\incimg}[2]{%
       \begin{figure}[h]
               \centering
               \includegraphics[scale=#2]{./figures/#1}
       \end{figure}
}

\title{\coursetitle\linebreak\lecturename}
\author{\\Cain Susko\\ 
           \\ \\ \\
      Queen's University 
    \\School of Computing\\} 

%=-=-=-=-=-title-=-=-=-=-=%
\newcommand{\lecturename}{The State, Urban Planning, and Politics}
\newcommand{\coursetitle}{Urban Planning}
%=-=-=-=-=-#####-=-=-=-=-=%

\begin{document}
\begin{titlepage}
        \maketitle
\end{titlepage}


\section*{Capitalism, The State, and Urban Planning}
Cities play an important role in our society by providing places to live and work.
The way a cities is planned is thus an important and contentious issue.
Because there is potentially alot to gain or lose in the field of governance--there
sometimes forms whats known as a `growth coalition'.

In the contemporary North American city, the state plays an important role in 
sustaining capitalist production. 
This can be seen in planning policy as the majority of decisions are `who gets what 
and why'

\paragraph{Growth Coalitions}
Growth coalitions are groups of business organizations and politicians who use their
money and influence to maintain their common business interests with little 
regard for the people living in the city. They tend to focus on land policy
in a way that promotes future land value growth as well as public-private partnerships.
Thus, if we can see how influential these groups are, we are forced to ask the same
question as previous classes: Cities For Whom?

\paragraph{}
An example is the HighLine in New York City, where an abandoned railway was turned 
into a park, and making it a worldwide tourist attraction. It brought alot of 
opportunities for investment; but it also disenfranchised and made the area 
un-liveable for the existing residents.

\section*{What is Planning? A Modern Profession}
Urban planning is the process of designing and managing change in the urban
environment. There is planning as a `transhistorical' practice that has existed 
as long as cities have; and planning in the context of the modern industrial 
and postindustrial city. 

\paragraph{Planning as a Modern Profession}
As the industrial revolution began and went on more people came to cities that were
never meant to hold such a large population, thus the job of a `planner' was born. 
This new profession began in Eurpoe as a modern, utopian, idealist way of thought
of the city.
\pagebreak

\paragraph{Haussman's Paris}
An example of planning in the industrial city is Hausmann's remaking of Paris. 
During the industrial revolution Paris was a dark and cramped place which was 
very helpful to urban revolutionaries blockading roads. It was so 
unpleasant in fact, Napolean III demanded that the situation be rectified and so 
Baron Haussman was tasked with redesigning Paris. He created long, expansive
boulevards and mandated a strict code for the appearance of buildings in 
central city. While this certainly improved the appearance of the city, it did
so without regard for the existing residents and plowed through many homes to make
way for the new roads and buildings.
\incimg{paris}{0.4}
\incimg{parismap}{0.6}

\paragraph{Howard's Garden City}
The idea of the garden city is to create sattelite cities outside of a major urban area.
These would be inexpensive properties built around a factory or industrial complex.
What is unique about the Garden City however, is that the people would share 
ownership of the land and `city'. This provided a safe and affordable place for
people to live--near a city but without the clamour. And in the country--without the 
idleness.
\incimg{gardencity}{0.4}

Unfourtunately, the radical elements like changing who owns the land and the social 
order were stripped from the implementations of the Garden City; resulting is merely,
very well designed Suburbs
\incimg{hampstead}{0.4}

\section*{Evolutions in Planning Thought}
Since the 1950's there have been many changes to how urban planning is thought of.
In the 50's, the scientific method was very pervelant in many professions; this
resulted in the planning practice becoming a practice of science.

\paragraph{The Rational Planning Model}
Edward Banfield introduced the Rational Planning Model. This promoted the idea of 
planning as an objective science.
\incimg{rationalplanning}{0.4}

This idea however, was challenged by the 60's.
\paragraph{Robert Moses}
The most infamous proprioter of Rational Planning is Robert Moses; who blew 
through whole neighborhoods with Highwaysin New York City with 
genuine and blatant disregard for people's wellbeing. Only focusing on
the fact that he was able to build the leviathan that was the Cross-Bronx Expressway.

\paragraph{Jane Jacobs}
Jacobs was the loudest opponent to Moses in NYC and rational planning in general.
She argued that cities are inherently complex and assuming that the simple, rational
solution is the best solution is incorrect. She says that people should be allowed
more involvement in urban planning.

\paragraph{Urban Renewal}
Urban Renewal was the term used by planners in the 60's to justify the demolishion
of so-called blighted areas. These blighted areas tented to be made up of 
minority communities which were chosen because of their ethnic makeup. An example
of urban renewal is the destruction of AfricVille in Halifax, N.S. There was
however, resitance similar to Jacobs' methods like conducting alternate surveys and
studies. Although, This did not help them in the same way it Jacobs and New York.

\paragraph{Real Public Input Planning Processes}
In 1969 the idea of public opinion was more important in the planning profession, as is
shown by Arnstien's Ladder of Participation.
\incimg{ladder}{0.4}

This was also done through the idea of the planner as an advocate for the
disenfranchised by communicating with the community. Generally, there was a shift
towards a more communicative approach to planning.

\paragraph{Indigenous Concerns}
Urban planning has bee used as a tool in Canada in order to illigitimize claims by 
indigenous peoples on the land being occupied by settlers.

\paragraph{Growth Oriented Planning}
Most cities are planned with growth in mind, however this way of thought is not 
sustainable and thus is not a way to move forward in the future.

\section*{Contemporary Planning Practice}
The primary form of planning in the modern world are zoning and official plans.

\paragraph{Zoning}
Zoning is the restrictions on land for what can be built there. Below is a zoning map 
of Kingston Ontario. Zoning is essentially the map for how people envision and imagine
the city to be
\incimg{zoning}{0.5}

There are however, mechanisms that allow a developer to building outside the zoning
requirements. 

\paragraph{Plans}
Official Pans and Secondary Plans are a more proactive than zoning, focusing on the
future growth of the city and policy framework to guide the city. A Secondary plan is 
one that is more detailed than the Official plan, but less than the Zoning plan. 

\paragraph{New Urbanism}
New urbanism draws on urban design interventions to promote walkable,
mixed-used neighbourhoods: “design principles produce a life that is well
worth living by providing places that enrich, uplift, and inspire the human spirit” 

\section*{Planning and the Metropolitan Region}
The regional scale of many urban problems does not always match up with existing
municipal boundaries. Drivers of Regionalism include:
\begin{itemize}
        \item Population Growth
        \item Download of Responsibilities
        \item The new international division of labour
\end{itemize}
There are many ways to address regionalism that we will discuss.

\paragraph{New Regionalism}
This is the idea that government organizations should cooperate as a region in order
to better plan and work together as a region. The problem with this idea however, is 
that these government organizations do not often work together an there arent 
neccisarily political means to do so. FUrthermore, they tend to compete with eachother
for investment. An example of a Regional city where many municipalities are 
within the same metopolitan region is Toronto Ontario 
\incimg{toronto}{0.5}

The deliniation of the region often shifts based on the identified problem or based
on institutional boundaries.

\paragraph{Annexation}
the way cities used to govern a region is through annexation be annexing areas that
are thought to be a part of the city. Below is the annexation map of Toronto from
1883 to 1967.
\incimg{annex}{0.5}

However, the regions that were annexed did not always want to be--fearing they would
lose what made them unique.

\paragraph{2 Tiered Systems}
In 1954 during the peak of suburban development, Toronto designed and implemented a 
2 tiered governmental system of Metro Toronto. This system allowed suburbs to still
manage things within its self but also be a part of the regional coordination of 
the Metro Toronto Area.
\incimg{2tier}{0.2}

\paragraph{Intermunicipal Agreements}
The idea of a intermunicipal agreements came with the neo-liberal consensus of `more
government bad' and so the intermunicipal agreement or special purpose body exists
in order to coordinate municipalities on a specific issue 
without restructuring government
\end{document}

