\documentclass[12pt]{book} 


\usepackage{amsmath}
\usepackage{graphicx}
\usepackage{import}
\usepackage{amsfonts}
\usepackage{booktabs}


\setlength{\parindent}{0em}  % sets auto indent at new paragraph to none

\newcommand{\incfig}[1]{%
        \import{./figures/}{#1.pdf_tex}
}

\newcommand{\incimg}[2]{%
       \begin{figure}[h]
               \centering
               \includegraphics[scale = #2]{./figures/#1}
       \end{figure}
}

\title{\coursetitle\linebreak\lecturename}
\author{\\Cain Susko\\ 
           \\ \\ \\
      Queen's University 
    \\School of Computing\\} 

%=-=-=-=-=-title-=-=-=-=-=%
\newcommand{\lecturename}{Developing Toronto Through Transit Oriented Development in Eastern Candada}
\newcommand{\coursetitle}{Urban Geography}
%=-=-=-=-=-#####-=-=-=-=-=%

\begin{document}
\begin{titlepage}
        \maketitle
\end{titlepage}


\section*{1 Introduction \& Summary} % ~336 words
\paragraph*{}
The need to move around the city is constant and ever-changing. As cities grow and grapple with the impacts of climate change, they are turning 
to new options for moving their citizens. This paper explores the proposed implementation of transit-oriented development in Ontario through the 
concept of Ordinary Cities concerning Critical Geography. Transit-Oriented Development–with the aid of precedents from other cities in Canada–is 
the path to catalyze the development of Complete Communities, which leads to opportunities for growth and revitalization in Toronto and the surrounding 
area.

\paragraph*{}
As Canadian cities continue to grow and the effects of climate change become ever more present in our daily lives, the way we travel within our cities 
has been reconsidered in recent years. Transit-oriented development is a new urban paradigm that has become popular in the past decade, although the 
concept has existed since the advent of mass transit within cities. Toronto recently commissioned a report on transit-oriented development; this report 
outlined how Toronto and the Greater Toronto Area could implement concepts within transit-oriented development in order to create stronger local 
economies and increase mobility within communities and the wider region.\cite{report} 

\paragraph*{}
These decisions were made taking into consideration the results of transit-oriented and mass transit developments around Eastern Canada. The REM project in Montr ́eal is a prime example; it exemplifies the inter-governmental cooperation needed to create projects where building a stronger community is at 
its centre. Additionally, the case study in Markham demonstrated how an implementation of all day GO Transit regional rail service could increase 
social mobility and foster economic growth. Furthermore, the report gives a prime example of sub-urban transit in Eastern Canada is the Hurontario Light Rail 
system which has provided a catalyst for densification within the satellite region. These examples exemplify the ways in which Toronto 
can use the experience and initiatives in ordinary cities in order to educate the decisions made by the city government to foster complete communities, and 
through these, economic growth.\cite{report}


\section*{2 Ordinary Cities Framework} % ~192 words
\paragraph*{}
The Ordinary cities framework states that the divide between cities seen as world class and those chich are
not should not be considered in defferent spaces and instead be observed as case studies for eahother in order to better undertand
the problems in both and implement solutions found in either city.\cite{ordinary} The approach is a post-colonial critique 
of the unfounded centralization on western cities within urban studies. The framework takes into account previously seperate liturature--drawing 
on the previous cosmopolitan approach to the city as well as case studies from cities including Rio Dejenero, Kuala Lumpur, and Johannesburg.
\cite{ordinary}

\paragraph*{}
The Ordinary Approach is Applicable to this issue as it relates to how the city has decided to inform themselves as they continue to
develop their urban community. The framework is esspecially suited to this region as there are many budding urban areas in the Greater
Golden Horseshoe and thus there are numerous examples of unique transit solutions suited for the Ontario region\cite{report}. The news story shines 
a light on the strides small communities in Canada are taking in terms of transit and how larger metropolitan areas can learn lessons 
from the successes--and failures.


\section*{3 Interperetation through the Framework} % ~552 words
\paragraph*{}
The Article I that is to be analyzed is by Daily Commercial News: 'Panel highlights the tools necessary to improve transit-oriented communities'.
This text highlights the plans being made for improving the Toronto \\region transportation.\cite{article}

\paragraph*{}
The article starts out by outlining a plan that has recently been published outlining the incentives for 
investing in Transit Oriented Development. The objective of the plan titles:\textit{Getting to Transit Oriented Communities}, is to foster
catalysts for community and economic growth within many areas in the Greater Toronto Area. This can be seen as a unique application 
of the Ordinar Cities framework as it seeks to not only apply the lessons learned from other cities to Toronto, but also smaller 
urban areas within the GTA as a platform for the future growth of toronto through transit oriented development.\cite{article}
The article then goes on to emphasise the implact that investment in transit oriented development has on surrounding communities and 
how it can often be the impetus for the strengthening of communities as well as their economies. These claims are backed up within the 
Ordinary Cities framework by multiple case studies in the afformentioned report. Highlighing the lessons learned with regards to intra-
governmental communication in particular. This attepmts to resolve Jennifer Robinson's critique of urban studies in her book, \textit{Ordinary Cities}
, which articulates that traditional urban scholars were too quick to segregate services, infrastructure, and planning of  different 
areas within a metropolitan region based on percieved economic stature or way of life.\cite{essay} The report does this by allowing the lessons learned 
through the planning of smaller projects in less populus cities in canada and even the Toronto Region in order to better inform the 
decisions made with data from the specific climate and region proposed infrastructure would be operating in.

\paragraph*{}
The article then outlined the key learning from this report, which was the importance of cities, governments, and communities 
to have a shared vision for the results they want to see from their planned transit oriented development. they based these findings
on 4 major case studies on Transit Oriented Develoment from Eastern Canada. These Infrastructure projects were the Science Centre Station 
as apart of the Eglington Crosstown LRT; the Markham Centre development which is now served by the GO Stouffville line; the Hurontario
Light rail project spurring Brampton Uptown Hurontario-Steeles corridor; and the REM regional rail system being built in Montr\'el.\cite{report}
Out of these case studies the REM in particular stood out as an exemplar of the level of engagement needed for intra-community work that
will result in efficient and useful transit oriented development. In contrast, the article also outlines an example from Brampton where 
there was a lack of intra-department communication as 8-foot side walks to improve the pedestrian space were designed from Downtown Brampton
without the input of the operations department, which lacks an 8-foot wide snow plow for such a road surface.\cite{article} 
This not only gives an example of how intra-government cooperation is important but is also gives an indication of how easy it is for oversight like 
lack of communication to lead to wider dissonance in cities around Ontario and Canada; where this interperetation is a prime example of how 
important a Ordinary Cities is to understanding urban issues.


\section*{4 Discussion \& Conclusion} % ~229 words
\paragraph*{}
Thus, there is clearly work to be done and solutions to the problems facing many Canadian cites within the ideas of Transit Oriented Development.
this can be seen in the lessons learned in the REM Project and the concerns brought up in many case studys including the Markham Go line and
Brampton Downtown Revitalizaton spurred on by Transit Oriented Development. these examples give firm precenence for the argument that transit oriented
development is a catalyst for development in areas that it is built and also shows one how using many frameworks, like the Ordinary Cities approch, 
allows for a better and deeper understanding of problems facing a city; like how the disjoint between government bodies related to the city can lead
to mismanagement and corruption, which only hurts the people of the city.\cite{article}

\paragraph*{}
The Ordinary Cities Framework allowed me to look beyond the monetary and political aspects of the infrastructure imporvement project and focus on the 
implications it has for the people living in the city and in particular, how actions taken by other cities in the past coud affect the city in the 
future. For example, Toronto could avoid problems between train lines by taking note of what happened in Markham between the
GO tracks and Freight Tracks by creating pre-emptive grade seperations, is is with what is happening in the current GO expansion in 
Toronto.\cite{report}
\begin{thebibliography}{1}
        \bibitem{ordinary}
        J. Robinson, \textit{Ordinary Cities : Between Modernity and Development}, London, Routledge, 2006

        \bibitem{article}
        D. Wall, \textit{Panel highlights the tools necessary to improve transit-oriented communities}, Canada, Construct Connect, jan 2022

        \bibitem{essay}
        J. C. Fraser, \textit{Globalization, Development and Ordinary Cities: A Review Essay}, JWSR, vol. 12, no. 1, pp. 189-197, feb 2006

        \bibitem{report}
        Future Of Infrasturcture Group, \textit{Getting to Transit Oriented Communities}, ULI Toronto, 2022
\end{thebibliography}
\end{document}
