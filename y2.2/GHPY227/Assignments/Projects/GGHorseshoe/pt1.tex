\documentclass[12pt]{book} 

\usepackage{amsmath}
\usepackage{graphicx}
\usepackage{import}
\usepackage{amsfonts}
\usepackage{booktabs}

\setlength{\parindent}{0em}  % sets auto indent at new paragraph to none

\newcommand{\incfig}[1]{%
        \import{./figures/}{#1.pdf_tex}
}

\newcommand{\incimg}[2]{%
       \begin{figure}[h]
               \centering
               \includegraphics[scale = #2]{./figures/#1}
       \end{figure}
}

\title{\coursetitle\linebreak\lecturename}
\author{\\Cain Susko\\ 
           \\ \\ \\
      Queen's University 
    \\School of Computing\\} 

%=-=-=-=-=-title-=-=-=-=-=%
\newcommand{\lecturename}{Urban Geography}
\newcommand{\coursetitle}{The Growth Plan for the Greater Golden Horseshoe\\Policy Analysis}
%=-=-=-=-=-#####-=-=-=-=-=%

\begin{document}
\begin{titlepage}
        \maketitle
\end{titlepage}


\section*{Plan Objectives}
\paragraph*{}
\textit{A Place to Grow: Growth Plan for the Greater Golden Horseshoe (also known as the GGH)} is the Government of Ontario's initiative
to support development and growth in the GGH such that economic prosperity, security, and a higher quality of life
can be achieved by individuals and enterprises in the GGH.

\paragraph*{}
The plan is the first of its kind to provide a framework for implementing the Government of Ontario's plan for growth 
which hopes to build stronger, more prosperous communities by better managing growth within the GGH. This Framework
is meant to be used long into the future in order to facilitate growth in both rural and urban communities in order to 
improve Ontario's towns and cities in the long term while taking into account factors that governments cannot control.

\paragraph*{}
The Government of Ontario's vision is for the GGH to continue to grow and mature into a global economic force as well as
continue to integrate into the surrounding region and become an access point for Canadians to the wider world.


\section*{Policy Mechanism}
\paragraph*{}
Section 2.2.4 of the growth plan: \textit{Transit Corridors and Station Areas} outlines policies governments in the Greater Golden Horseshoe
could enact in order to better incentivize Transit Oriented Development. 

\paragraph*{}
The policies outlined in this section focus on the designation of transit corridors and station areas in order to make it easier for 
denser development around these high-demand but often neglected areas. Transit corridors within the framework are intended to be outlined
within Official Plans for the GGH and it's constituent regions. Transit Station Areas are delineated by all levels of government as the
concerns of each area are unique and require those at the top of government and those who live in thew communities to be involved. These areas
are to be designated with minimum density requirements. These are:
\begin{itemize}
        \item 200 residents and jobs total per hectare for areas served by the Subway
        \item 160 residents and jobs total per hectare for areas served by light rail or bus rapid transit (BRT)
        \item 150 residents and jobs combined per hectare for ares served by GO regional rail.
\end{itemize}

\paragraph*{}
In Addition to these minimum requirements the framework outlined by the plan suggests policies which prohibit development which
would inhibit the area reaching it's required density. The framework also suggests that transit station areas are to be supported by
zoning which allows both multi-modal transportation as well as medium-density mixed-use developments to foster economic and
social mobility. Finally, in order to facilitate future growth, the plan states that land adjacent to existing or planned transit corridors 
should be pre-allocated to transit oriented development or transit infrastructure.

\section*{Sources}
Ontario Legislative Assembly. (2020, August). A place to grow - premier of ontario. \textit{A Place To Grow: A Growth Plan For The Greater Golden Horseshoe.} Retrieved March 19, 2022, from https://files.ontario.ca/mmah-place-to-grow-office-consolidation-en-2020-08-28.pdf


\end{document}

