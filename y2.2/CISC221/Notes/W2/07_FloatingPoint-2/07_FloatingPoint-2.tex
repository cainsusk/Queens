\documentclass[12pt]{book} 

\usepackage{amsmath}

\newcommand{\classID}{Floating Points 2}
\newcommand{\coursename}{Computer Architecture}

\begin{document}
\date{}
\setlength{\parindent}{0em}  % sets auto indent at new paragraph to none

\title{\coursename\\\classID}

\author{\\ \\ Cain Susko\\\today \\ \\ \\ \\ \\
        Queen's University \\School of Computing} 
 

\maketitle
\pagebreak

\section*{Floating Point Operations}
there are 2 operations with floating points:
\begin{align*}
        x+_f y &= round(x+y)\\
        x \times_f y &= round\left( x+y \right) 
.\end{align*} 

We first compute the exact result and then make it fit into the precision of
        a given $w$-bit numbers.
When rounding we default to rounding to the closest \textbf{whole} number.
If the number is equally distant from either whole number, round to the \textbf{even}
        number.\\
\[2.50 = 2\]
When dealing with binary numbers, if the value to in the middle of 2 possible values
        then round so that the least significant digit is even.
\[2.\frac{2}{32}=10.00011_2 \rightarrow 10.00_2 = 2\]
\[2.\frac{3}{16}=10.00110_2 \rightarrow 10.01 = 2.\frac{1}{4}\]
note the decimal numbers are in the form int.frac where frac is a representation of
        what the number past the decimal is.
\subsection*{Addition}
the exact result of adding 2 floating point numbers in the form of IEEE float is as follows:
\[(-1)^{s_1} M_1 *2^{E_1} + (-1)^{s_2} M_2 *2^{E_2}= (-1)^sM*2^E\]
such that $s$ and  $M$ are the results of signed align and add as well as
        $E$ is equal to $E_1$.
We must also fix some things about his sum:
\begin{enumerate}
        \item if $M \geq 2$, shift $M$ right, increment $E$
        \item if $M<0$ shift  $M$ left  $k$ positions, decrement  $E$ by $k$.
        \item overflow if $E$ is out of range
        \item round  $M$ to fit  $frac$ precision.
\end{enumerate}

This operation is \textbf{Closed, Commutative, not Associative, almost Invertible, 
        almost Monotonous}
the almost is because of infinity and NaN.

\pagebreak

\subsection*{Multiplication}
multiplying 2 IEEE floating point numbers is as follows:
\[(-1)^{s_1} M_1 *2^{E_1} \times (-1)^{s_2} M_2 *2^{E_2}= (-1)^sM*2^E\]

Such that the sign value $s=s_1 * s_2$, the significand $M=M_1\times M_2$, 
        and  $E=E_1+E_2$.
The conditions where we need to fix things with the product are as follows:
\begin{enumerate}
        \item if $M \geq 2$, shift $M$ right, increment $E$
        \item if $E$ is out of range, overflow
        \item round  $M$ to fit $frac$ precision.
\end{enumerate}

When implementing floating point multiplication the biggest job is multiplying 
        significands.\\
This operation is \textbf{Closed, Commutative, not Associative, Invertible,
        not Distributive over addition, almost Monotonous}.
The almost is because of infinities and NaN.

\section*{Float in C}
a float in C Guarantees 2 levels of precision:
\begin{enumerate}
        \item float-single precision
        \item double-double presision
\end{enumerate}
the conversions between these types and the int type are as follows:\\
\begin{itemize}
        \item double$\vee$float $\rightarrow$ int\\truncated  
                fractional part and rounds towards 0\\
        \item int $\rightarrow$double\\is an exact conversion as long as int has  
                word size.\\
        \item int $\rightarrow$float\\will round according to rounding mode
\end{itemize}

\section*{Summary}
casting signed to unsigned integers results in their bit patterns being maintained
        but reinterpereted.
This can have the unexpected effect of adding or subtracting $2^w$\\
When \textbf{Expanding} the general rules are:
\begin{itemize}
        \item Unsigned: Zeros added
        \item Signed: sign extension (add s bit)
        \item Both yeild expected result
\end{itemize}

Similarly, when \textbf{Truncating} the general rules are:
\begin{itemize}
        \item Unsigned$\vee$Signed: bits are truncated. Result reinterpreted
        \item  Unsigned: mod operation
        \item Signed: similar to mod operation
        \item for small numbers: yields expected behaviour  
\end{itemize}

Rules for \textbf{addition}
\begin{itemize}
        \item Unsigned/Signed: normal addition followed by truncate, 
                same operation on bit level.
        \item Unsigned: addition mod $2^w$
        \item Signed: modified addition mod  $2^w$ such that the result is in the  
                proper range
\end{itemize}

Rules for \textbf{multiplication} 
\begin{itemize}
        \item Unsigned/Signed: normal multiplication followed by truncate, same
                operation at bit level.
        \item Unsigned: multiplication mod $2^w$.
        \item Signed: multiplication mod  $2^w$ within proper range.
\end{itemize}
\end{document}
