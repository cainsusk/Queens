\documentclass[12pt]{book} 

\usepackage{amsmath}

\newcommand{\classID}{Integer Arithmatic}

\begin{document}
\date{}
\setlength{\parindent}{0em}  % sets auto indent at new paragrahp to none

\title{Computer Architecture\\\classID}

\author{\\ \\ Cain Susko\\\today \\ \\ \\ \\ \\
        Queen's University \\School of Computing} 
 

\maketitle
\pagebreak


\section*{Addition}
\subsection*{Unsigned Addition}
when adding $w$-bit numbers, the sum may require more than $w$ bits to represent,
        thus, we must apply modular arithmatic to the sum.\\
a \textbf{true} sum is one that \textit{does not} need to use modular arithmatic. an \textbf{untrue} sum does.
\[sum = x+^{u}_{w}y \Rightarrow sum = x+y\text{ mod } 2^w\]

\paragraph{}
Thus, any 4 bit integers $x$ and $y$ can compute the true sum $x+y$. this value increases linearly with $x$ and $y$ and if one were
        to graph this in 3D with axis $x$, $y$, $x+y$ one would get a plane.\\
If one were to graph these values again, but where $x+y \geq 2^4$. in other words, the sum cannot strictly be represented in 4 bits (untrue sums). one would see
        many identical planes representing the 'wrap around' effect of modular arithmatic.

\paragraph{}
Modular addition forms an \textbf{Albelian Group} which means it is closed, communative, and associative, with an identity of 0 and 
        inverses for all operable elements.\\
respectively they are:
\begin{align*}
        o \leq x &+^{u}_{w}y\leq 2^w -1\\
        x +^{u}_{w}y &= y +^{u}_{w}x\\
        (x +^{u}_{w}y)+^{u}_{w}z &= x +^{u}_{w}(y +^{u}_{w}z)\\
        x+^{u}_{w}0 &= x\\
        2^w&-x
\end{align*}

\pagebreak 


\subsection*{Signed Addition}
this is also known as 2's Complement Addition\\
given the following equation, the classifications for the result are:
\[x+^{t}_{w}y\]
\begin{align*}
        &2^{w-1} \leq x+y &\textbf{Positive Overflow}\\
        &x+y < 2^{w-1} &\textbf{Negative Overflow}\\
        &-2^{w-1} \leq x+y \leq 2^{w-1} &\textbf{Normal}\\
\end{align*}
note that $+^{t}_{w}$ and $+^{u}_{w}$, which are our mod operators, have the same bit level behaviour (in C).
a true sum requires that there are $w+1$ bits to use\\
similar to Unsigned addition, graphing $x$, $y$ and $x+^{t}_{w}y$ gives us many identical planes representing the range of 
$x+^{t}_{w}y$.

\section*{Complement and Incrament}
the claim is this:\\
for 2's compement numbers, the following holds:
\[\neg x+1 = -x\]
thus the \textbf{complement} of a number is:
\begin{align*}
        &x&=15213        &=00111011 01101101\\
        &\neg x&=-15214  &=11000100 10010010\\
        &\neg x+1&=-15213&=11000100 10010011
\end{align*}
thus $\neg x+1$ is the complement of $x$.

\pagebreak


\section*{Multiplication}
multiplication can be done with both Signed and Unsigend integers. 
Just the same as in addition, we can have a value that is cannot be represented by $w$ bits.
\begin{align*}
        &\textbf{Unsigned}      &0 \leq x*y\leq (2^w -1)^2 = 2^{2w} - 2^{x+1}+1\\
        &\textbf{Signed +}      &x*y \geq (-2^{w-1}*2^{w-1}-1) = -2^{2w-2}+w^{w-1}\\
        &\textbf{Signed -}      &x*y\leq (-2^{w-1})^2 = 2^{2w-2}
\end{align*}
$w$ would have to change after each multiplication in order to maintain accuracy. otherwise we just use modular arithmetic
\[x*^{t}_{w}y\]
the overflow from this operator will be reported by the system unless amother solution is being used ie. increasing $w$


\section*{Power-of-2 by shifting}
we can multiply ints by powers of 2 using the shift operator. 
For example:
\[x << k == x*2^k\]
once we have the result for this operation, we then apply the same modular arithmatic for Signed or Unsigned ints respectively.

\subsection*{Shift Multiply}
we can also multiply by numbers that are not powers of 2 by represented them through multiple shift operations:
\[(x<<5)-(x<<3) == x*24\]
thus we can use this technique to multiply by any number//
this is useful as most computers can do shifting and addition/subtraction much faster than multiplication.
most compilers use this technique.

\subsection*{Shift Division}
division can be emulated in the same way with the other shift operator. there is also a floor operator as we are
        only wokring with ints. we \textbf{always round down}
\[x>>k == \lfloor x/2^k\rfloor \]
note that this uses Logical shift


\section*{Summary}
computer arithmetic is bound by the number of bits that we are using $w$ and we must account for any overflow either by increasing $w$
or by implementing modular arithmetic.

\end{document}
