\documentclass[12pt]{book} 

\usepackage{amsmath}
\usepackage{graphicx}
\usepackage{import}
\usepackage{amsfonts}
\usepackage{booktabs}

\setlength{\parindent}{0em}  % sets auto indent at new paragraph to none

\newcommand{\incfig}[1]{%
        \import{./figures/}{#1.pdf_tex}
}

\newcommand{\incimg}[2]{%
       \begin{figure}[h]
               \centering
               \includegraphics[scale = #2]{./figures/#1}
       \end{figure}
}

\title{\coursetitle\linebreak\lecturename}
\author{\\Cain Susko\\ 
           \\ \\ \\
      Queen's University 
    \\School of Computing\\} 

%=-=-=-=-=-title-=-=-=-=-=%
\newcommand{\lecturename}{Dont Cares in Digital Circiuts}
\newcommand{\coursetitle}{Computer Architecture}
%=-=-=-=-=-#####-=-=-=-=-=%

\begin{document}
\begin{titlepage}
        \maketitle
\end{titlepage}


\section*{Dont Cares}
Many circuits have multiple outputs, each 
of which computes a separate Boolean 
function of the inputs  
When there are strong correlation among 
the outputs, the truth table can often be 
simplified with “don’t care” as certain 
input values don’t matter in some cases
\incimg{circuit}{0.5}

\incimg{dontcare}{0.7}

\pagebreak

\section*{Karnaugh Map}
A karnagh map is a way of simplifying boolean expressions. it was developed by Maurice Karnaugh of Bell Labs in the 1950's. the idea of the k-map is
that by arranging the minterms on a grid, it makes combinable terms easier to recognize.
\incimg{karnaugh}{0.5}
Take another example. Note, the C, AB represent true or false values of variables, one has to fill
in the karnaugh map with the corresponding boolean values.
\incimg{karnaugh2}{0.5}
We can also use up to 4 variables from a boolean equation

Also, note that the table is `connected' such that the going past the top goes to the bottom, and going too far left puts you as far right.

Areas that are able to be simplified are ones that are consecutive and have the same bit (dont cares are counted as 1, also, they are represented as $x$)
Note, when combining variables in the boxes, the variables that are kept from the simplification are the ones that stay the same within the box.
\end{document}

