\documentclass[12pt]{book} 

\usepackage{amsmath}
\usepackage{graphicx}
\usepackage{import}
\usepackage{amsfonts}
\usepackage{booktabs}

\setlength{\parindent}{0em}  % sets auto indent at new paragraph to none

\newcommand{\incfig}[1]{%
        \import{./figures/}{#1.pdf_tex}
}

\newcommand{\incimg}[2]{%
       \begin{figure}[h]
               \centering
               \includegraphics[scale = #2]{./figures/#1}
       \end{figure}
}

\title{\coursetitle\linebreak\lecturename}
\author{\\Cain Susko\\ 
           \\ \\ \\
      Queen's University 
    \\School of Computing\\} 

%=-=-=-=-=-title-=-=-=-=-=%
\newcommand{\lecturename}{Digital Ciruits}
\newcommand{\coursetitle}{Software Specifications}
%=-=-=-=-=-#####-=-=-=-=-=%

\begin{document}
\begin{titlepage}
        \maketitle
\end{titlepage}


\section*{Digital circiuts}
digital circiuts are ophysical login chips that can be represented with boolean logic.

there are 2 types of digital circiuts:
\begin{itemize}
        \item combinational circiuts
        \item sequential circiuts
\end{itemize}

\section*{Combinational Circuits}
the functional specifications of a combinational circiut expresses the output values in terms of the input values.
The timing specifications are consists of a lower and upper bound on the delay from input to output.
a given combinational logic can be represented in many different ways.
in diagrams, represent a combinational circiut as $CL$ and use a line with a slash through it for a bus $\not --$

\paragraph{rules}
in a combinatorial circuit, there are a few limitations as to what kind of circuit is combinational.
\begin{itemize}
        \item every circuit element is itslef, combinational
        \item every node in the circiuts is either designated as an input or connected to a output
        \item there are no cyclic paths
\end{itemize}

\paragraph{creation}
to make a combinational circiut one must:
\begin{itemize}
        \item create a truth table with columns for input and output, and a row for each possible combination of inputs. define what output should be.
        \item from the table, create a boolean function for each output
        \item create a logic network from boolean expressions using logic gates in order to represent the boolena function.
\end{itemize}

\subsection*{Example}
given the following truth table, find the sum of product form:
\incimg{truthtable}{0.5}
where:
\[Y = \neg A \neg B \neg C + A \neg B \neg C + A \neg B C\]
\end{document}
or, in other terms: $Y = \Sigma(0,4,3)$.

the final step is to find the visual circuit representation of the boolean logic, like so:
\incimg{circiut}{0.5}

\section*{Boolean Algebra}
Boolean algebra is based on a set of axioms that we assume are correct.
From these axioms, we prove all the theorums of boolean algebra.
\incimg{bool1}{0.5}
\incimg{bool2}{0.5}

\paragraph{DeMorgans Law}
also known as push the bubble (apparently ?). there are many ways of writing demorgans law (is is the case with combinational logic)
Either as NAND or NOR
\incimg{demorgan}{0.5}

\paragraph{Simplifying Equations}
These are common tools that we can use to simplify boolean equation. Foe example:
\incimg{ex1}{0.5}
\incimg{ex2}{0.5}

the final visual result of simplifications like this can be like:
\incimg{ex3}{0.5}

