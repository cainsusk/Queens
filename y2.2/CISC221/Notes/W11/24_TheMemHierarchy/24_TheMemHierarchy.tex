\documentclass[12pt]{book} 

\usepackage{amsmath}
\usepackage{graphicx}
\usepackage{import}
\usepackage{amsfonts}
\usepackage{booktabs}

\setlength{\parindent}{0em}  % sets auto indent at new paragraph to none

\newcommand{\incfig}[1]{%
        \import{./figures/}{#1.pdf_tex}
}

\newcommand{\incimg}[2]{%
       \begin{figure}[h]
               \centering
               \includegraphics[scale = #2]{./figures/#1}
       \end{figure}
}

\title{\coursetitle\linebreak\lecturename}
\author{\\Cain Susko\\ 
           \\ \\ \\
      Queen's University 
    \\School of Computing\\} 

%=-=-=-=-=-title-=-=-=-=-=%
\newcommand{\lecturename}{The Memory Heirarchy}
\newcommand{\coursetitle}{Computer Architecture}
%=-=-=-=-=-#####-=-=-=-=-=%

\begin{document}
\begin{titlepage}
        \maketitle
\end{titlepage}


\section*{The Hierarchy}
different classes of memory are segregated by their speed of access,
their relative speeds can be visualized on a graph like so:
\incimg{memTri}{0.7}

Faster Memory classes are more expensive in resources, and thus are mostly
used in smaller quantities. (in addition to them being physically smaller
than the lower classes of chips)

\section*{Random-Access Memory}
RAM are traditionally chips who's basic storage unit was a cell. Multiple cells
of the word form a super cell, which id id'd by a single address. Many
RAM chips then make up the Computers memory. This is in contrast to Sequential 
Access Memory which can only be accessed by iterating a 
pointer along an array.
\pagebreak

\paragraph{RAM Organization}
the chip is normally organized with a 2D array like so; where the array 
has the dimensions of $d \times w$, where $dw$ are the total bits organized as
supercells of size $w$ bits.
\incimg{RAM}{0.7}

There are 4 bits for the access address because there is needed 2 bits for the 
row address and 2 bits for the column address. When a row is addressed, it is 
then saved into the RAM chip's internal row buffer. Then, then the column is 
addresses, the specified column within the row is taken out and returned. When 
writing this is the exact same thing but backwards. This is very fast as the
row can be copied with parallel circuits. This Process can also be scaled up 
with many (8) $8 \times 8$ RAM chips. if we had 64 bit word across this \textbf{
memory module}, we would copy into the memory controller buffer.
\incimg{memMod}{0.5}

\paragraph{SRAM v. DRAM}
there are 2 types of RAM, SRAM and DRAM. this first is Static RAM while the 
second is Dynamic RAM. Static RAM's are fast and costly while Dynamic RAM's
are slow and less costly, with only 1 transistor.
\incimg{ramvram}{0.5}

All of this memory is what is known as Volatile Memory, which means that
if the power were to be fully shut off on the computer, the data stored in these
chips would be lost, as, they require power to retain information.
This requires what is known as \textbf{non-volatile memory} 

\section*{Non-Volatile Memory}
There are a few types of memory that are non-volatile, meaning they do not lose
their data once power is cut to them.
\begin{itemize}
        \item Read-Only Memory (ROM): normally programmed during production of
                chip
        \item Programmable ROM (PROM): can be programmed once
        \item Erasable PROM (EPROM): can be bulk erased (commonly with UV or 
                x-rays)
        \item Electric EPROM (EEPROM): electronically erasable EPROM
        \item Flash Memory: EEPROM(s) with partial (block level)
                erase capability. These wear out after around 100,000 erasures.
\end{itemize}

These are used for Firmware programs stored in a ROM (BIOS, Controllers for Disks
, network cards, graphics accelerators, security subsystems, ...) or also Solid
State Disks.

\section*{Traditional Bus Structure}
A bus is a collection of parallel wires that carry addresses, data, and control
signals. A bus is shared between multiple devices, connecting them. The most 
important use of a bus is connecting the CPU to Memory in order to read and
write data for executing programs.
\incimg{bus}{0.5}

\paragraph{Reading}
When the CPU is \textbf{reading} memory the transaction goes as follows:
\begin{enumerate}
        \item The CPU places an address $A$ onto the memory bus - 
        \item The Main memory reads $A$ from the memory bi=us, and retrieves the 
                word $x$ at address $A$, and places $x$ on the bus 
        \item Lastly, the CPU reads the word $x$ from the bus and copies it into
                register \texttt{\%rax}
\end{enumerate}
Note: the assembly operation for this would be: \texttt{movq A, \%rax}

\paragraph{Writing}
When a CPU is \textbf{writing} to memory, the operation goes as follows:
\begin{enumerate}
        \item The CPU places the address $A$ on a bus, the Main memory then 
                reads it and waits for the corresponding data word to arrive.
        \item The CPU then places the data word $y$ on the bus
        \item Finally, the Main memory reads the data word $y$ from the bus and
                stores it at address $A$
\end{enumerate}
Note: the assembly operation for this would be: \texttt{movq \%rax, A}
\pagebreak

\section*{Hard Disk Drive}
If one were to open up a disk drive in their computer, it would look like this:
\incimg{disk}{0.5}
The spindle, in combination with the rotating disk, finds parts (or sectors) of
memory. This can be seen like so
\incimg{diskMem}{0.5}

Many sectors make a Track, and many tracks make a cylinder, which can be 
visualized as many disks stacked on top of each other:
\incimg{disks}{0.5}

\paragraph{Disk Memory}
The memory that can be stored in a disk can be using:
\begin{itemize}
        \item track density (bits/in)
        \item recording density (tracks/in)
        \item areal density (bits/$\text{in}^2$)
\end{itemize}

Note: it should be considered that because the disk is well, a disk, the diameter
gets smaller as the the radius gets smaller, meaning less data can be stored
closer to the middle of the disk. Therefore, disks are partitioned as 
such:
\incimg{diskPart}{0.7}

Therefore, the calculation for the capacity for a disk would be:
{\scriptsize
\[C = (bytes/sector)\times(avg.sectors/track)\times(tracks/surface)\times(surfaces/platter)\times(platters/disk)\]
}

\paragraph{Disk Operations}
Disk Reading and writing are carried out by the rotating disk and arm. when
there are multiple disks, there are multiple arms that all work in unison, and
all the disks are on one rotating spindle.

\incimg{diskOps}{0.7}




\end{document}

