\documentclass[12pt]{book} 

\usepackage{amsmath}
\usepackage{graphicx}
\usepackage{import}
\usepackage{amsfonts}
\usepackage{booktabs}

\setlength{\parindent}{0em}  % sets auto indent at new paragraph to none

\newcommand{\incfig}[1]{%
        \import{./figures/}{#1.pdf_tex}
}

\newcommand{\incimg}[2]{%
       \begin{figure}[h]
               \centering
               \includegraphics[scale = #2]{./figures/#1}
       \end{figure}
}

\title{\coursetitle\linebreak\lecturename}
\author{\\Cain Susko\\ 
           \\ \\ \\
      Queen's University 
    \\School of Computing\\} 

%=-=-=-=-=-title-=-=-=-=-=%
\newcommand{\lecturename}{Sequential Circiuts}
\newcommand{\coursetitle}{Computer Architecture}
%=-=-=-=-=-#####-=-=-=-=-=%

\begin{document}
\begin{titlepage}
        \maketitle
\end{titlepage}


\section*{Sequential Curcuits}
within a sequential circuit there are: 
\begin{itemize}
        \item Outputs depend on the current and past values of input
        \item this implies that the use of some type of memory device to store the effects of past 
                inputs
        \item It takes a sequence of inputs and generates a sequence of outputs
\end{itemize}
A sequential logic might remember previous inputs or it may just distill the previous inputs into
states to remeber.
Sequential circuits are widely used to implement computing and storage components of computer
systems.

\paragraph{Devices}
there are different devices we can implement within sequential circuits:
\begin{itemize}
        \item latches and flip-flops
        \item building blocks of sequential logic circuits - register, counter, shift register
        \item memory array and register file
\end{itemize}
these can be used to implement sometthing like a finite state machine

\section*{Single Bit Memory}
to have a on or off state, we must have what is known as a Bi-Stable Device. This is
the fundamental building block of memory. This device must be stable at one of the 2 states,
and must require an input to transition state. Examples of this device include a light switch
or an inverted pendulum (see below)
\incimg{pendulum}{0.5}
Another example is a coin flipping, where there are 2 states and to change the state one must pick
up and flip the coin. with many coins we could store more than $number of coins^2$
states as each coin is bi-stable.

\subsection*{Cross-Coupled Inverter Pair}
Within circuits, we can represent a bi-stable device using cross-coupled inverter pairs:
\incimg{crossCpl}{0.5}

However, Once the state is set it cannot be changed.

\subsection*{SR Latch}
The SR Latch is the actual way we implement a bi-stable device within circuits. The SR latch 
stores a single bit, and it;s value can be set by the \\\texttt{S} (set) and \texttt{R} (reset)
terminals. The disign is derived from the Cross coupled inverter pair.
\incimg{srlatch}{0.5}

\subsection*{D Latch}
A D (data) Latch is a addition to the SR latch and has:
\begin{itemize}
        \item a \texttt{D} (data) input
        \item a \texttt{CLK} (clock) input
\end{itemize}
these are to get inuput and control the timing of the input.
\incimg{dlatch}{0.5}

the \texttt{CLK} is a sequence on pulses such that it passes the value in \texttt{D} at a high 
voltage (at a specified interval). 

\subsection*{Flip Flop}
A Flip Flop stores a single bit of information. It has a \texttt{D} (data) and \texttt{CLK}
(timing).  It will only accept the value in \texttt{D} at the high voltage (rising edge)
of the \texttt{CLK} while `remembering' its state at all other times.

It is implemented with 2 back to back D latches--specifically utilizing the delay intorduced
by the \texttt{CLK} inverter (the \texttt{not} after the \texttt{CLK}):
\incimg{flipflop}{0.7}

The flip flop is going to remeber things and then is going to be able to execute things as the
transitions within the flip flop.
a D flip flop sybmol is:
\incimg{dflipsym}{0.5}

The D flip flop can be used to implement state machines. We also often use \texttt{S, S'} to 
label the current and next state. Some computers also have a `watchdog' which resets the 
D flip flop to not look at the current state and only look at the future.

\section*{Circuits}
Now, with these devices there are many circuits we can make.
\pagebreak

\paragraph{Registers}
An $N$-bit register is a bank of $N$ flip flops sharing a common \texttt{CLK}. Registers
are the key building block of sequential circuits.
\incimg{register}{0.7}

\paragraph{Counter}
A counter is a very important device within computers. An $N$ bit binary counter is a sequentail
circuit with:
\begin{itemize}
        \item clock and reset inputs
        \item an $N$-bit input
\end{itemize}
\incimg{clock}{1}

The counter implements an Adder, and a D Latch.
\pagebreak

\paragraph{Shift Register}
A shift register takes an input and shifts it by a given ammount. An $N$-bit regiser
has:
\begin{itemize}
        \item a serial input \texttt{$S_{in}$}
        \item a serial output \texttt{$S_{out}$}
        \item $N$ parallel outputs \texttt{$Q_{N-1:0}$}
\end{itemize}
on the risiing edge (high voltage) of the clock, a new bit is shifted in from $S_{in}$ and all
the subsequent contents are shifted forward. The last bit in the shift register is 
then available in $S_{out}$
\incimg{shiftReg}{0.8}



\end{document}

