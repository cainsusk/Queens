\documentclass[12pt]{book} 

\usepackage{amsmath}
\usepackage{graphicx}
\usepackage{import}

\setlength{\parindent}{0em}  % sets auto indent at new paragraph to none

\newcommand{\incfig}[1]{%
    \import{./figures/}{#1.pdf_tex}
}

\title{\coursetitle\linebreak\lecturename}
\author{\\Cain Susko\\ 
           \\ \\ \\
      Queen's University 
    \\School of Computing\\} 

%=-=-=-=-=-title-=-=-=-=-=%
\newcommand{\lecturename}{Numbers}
\newcommand{\coursetitle}{Computer Architecture}
%=-=-=-=-=-#####-=-=-=-=-=%

\begin{document}
\begin{titlepage}
        \maketitle
\end{titlepage}


\section*{Question 1}
Encode the following decimal numbers with 8-bit two’s complement
binary, or indicate that number would overflow the range:
\begin{enumerate}
        \item  $49_{10}$
        \item  $-31_{10}$
        \item  $120_{10}$
        \item  $-128_{10}$
        \item  $128_{10}$
\end{enumerate}

\subsection*{Answer}
A 2's complement number is a number encoded in $w$ bits with a range of: 
\[(-2^{w-1}, 2^{w-1}]\]

Where the most significant digit is the sign and the remaining $w-1$ bits represent the actual number.  
\begin{enumerate}
        \item 00110001
        \item 11100001
        \item 01111000
        \item 10000000
        \item Overflow 
\end{enumerate}
\pagebreak


\section*{Question 2}
Around 250 B.C., the Greek mathematician Archimedes proved that:
\[\frac{223}{71}<\pi<\frac{22}{7}\]
Had he had access to a computer and the standard library \texttt{<math.h>} he would have been able to 
        determine that the single precision floating-point approximation of $\pi$ has the hexadecimal
        representation:
        \[\texttt{0x40490FDB}\]
Of course, all of these are just approximations since $\pi$ is not rational.
\begin{enumerate}
        \item What is the Fractional binary number denoted by this floating point value?
        \item What is the Fractional binary number representation of $\frac{22}{7}$?
        \item At what bit positions relative to the binary point do these two approximations of $\pi$ diverge?
\end{enumerate}

\subsection*{Answer}
This question both requires the knowledge of converting hexadecimal to binary as well as converting binary to integer.
Furthermore one must also know how to read encoded floating point binary.
If we have a Fractional binary number with $w$ bits, a digit at position $i$ relative to the ones position represents
        the value  $2^i$. 
Given a $w$-bit binary string which represents the value  $x$, the Fractional decimal representation $N$ of this binary string is
\[N = \frac{x}{2^w}\]

\begin{enumerate}
        \item 01000000010010010000111111011011
        \item 01000000010010010010010010010010
        \item they differ at the $10^{th}$ position from the binary decimal (if counting up from 1)

\end{enumerate}
\end{document}

