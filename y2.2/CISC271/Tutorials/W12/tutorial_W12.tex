\documentclass[12pt]{book} 

\usepackage{amsmath}
\usepackage{graphicx}
\usepackage{import}
\usepackage{amsfonts}
\usepackage{booktabs}

\setlength{\parindent}{0em}  % sets auto indent at new paragraph to none

\newcommand{\incfig}[1]{%
        \import{./figures/}{#1.pdf_tex}
}

\newcommand{\incimg}[2]{%
       \begin{figure}[h]
               \centering
               \includegraphics[scale = #2]{./figures/#1}
       \end{figure}
}

\title{\coursetitle\linebreak\lecturename}
\author{\\Cain Susko\\ 
           \\ \\ \\
      Queen's University 
    \\School of Computing\\} 

%=-=-=-=-=-title-=-=-=-=-=%
\newcommand{\lecturename}{Week 12 Tutorial}
\newcommand{\coursetitle}{Linear Data Analysis}
%=-=-=-=-=-#####-=-=-=-=-=%

\begin{document}
\begin{titlepage}
        \maketitle
\end{titlepage}


\section*{Degree Laplacian Matrix}
you can calculate the Laplacian Matrix of a matrix \texttt{A} in matlab using:
\[\texttt{L = diag(sum(Amat, 2)) - Amat}\]

\section*{Sort}
we can sort the Eigenvectors from largest to smallest using:
\[\texttt{[lvec lx] = sort(EvecRaw, `descend')}\]

we can then sort the matrix of eigen-values using the sorted 
indicies (\texttt{lx}) such that:
\[\texttt{eMat = eMat(:, lx)}\]

such that we essentially permute eMat to be sorted.

\section*{Adjacency Matrix}
we can calculate an adjacency matrix without a loop by doing:
\begin{verbatim}
        % set A(i,j) to 1
        A(elist(:, 2) + (elist(:, 1)-1)*mA) = 1;
        % symmetric, handle edge duplicates
        A = (A+A') > 0;
\end{verbatim}

\end{document}

