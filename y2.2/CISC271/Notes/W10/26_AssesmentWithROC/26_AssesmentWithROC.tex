\documentclass[12pt]{book} 

\usepackage{amsmath}
\usepackage{graphicx}
\usepackage{import}
\usepackage{amsfonts}
\usepackage{booktabs}

\setlength{\parindent}{0em}  % sets auto indent at new paragraph to none

\newcommand{\incfig}[1]{%
        \import{./figures/}{#1.pdf_tex}
}

\newcommand{\incimg}[2]{%
       \begin{figure}[h]
               \centering
               \includegraphics[scale = #2]{./figures/#1}
       \end{figure}
}

\title{\coursetitle\linebreak\lecturename}
\author{\\Cain Susko\\ 
           \\ \\ \\
      Queen's University 
    \\School of Computing\\} 

%=-=-=-=-=-title-=-=-=-=-=%
\newcommand{\lecturename}{Classification - Assesment With ROC Curve}
\newcommand{\coursetitle}{Linear Data Analysis}
%=-=-=-=-=-#####-=-=-=-=-=%

\begin{document}
\begin{titlepage}
        \maketitle
\end{titlepage}


\section*{a Reciever Operator Charaterisitc}
Reciever Operator Charaterisitc (ROC) is a curve for scoring data. We do this by finding the 
Area Under the Curve or (AUC).

An early use of the ROC included radar threat detection in wartime. The Reciever is the Operator
of the radar; the Operator is the human who interpereted the display; and the Characteristic is
the evaluation of human performance.


\section*{b ROC And The Confusion Matrix}
This section relates the confusion matrix to ROC.
we have encountered an Absolute confusion matrix:
\begin{table}[h]
        \centering
\begin{tabular}{@{}lll@{}}
\toprule
                        & +1 & -1 \\ \midrule
\multicolumn{1}{l|}{+1} & TP & FN  \\
\multicolumn{1}{l|}{-1} & FP & TN   \\ \bottomrule
\end{tabular}
\end{table}

but we can also make a \textbf{realtive} Confusion Matrix.
\begin{table}[h]
        \centering
\begin{tabular}{@{}lll@{}}
\toprule
                        & +1 & -1  \\ \midrule
\multicolumn{1}{l|}{+1} &TPR & FNR \\
\multicolumn{1}{l|}{-1} &FPR & TNR \\ \bottomrule
\end{tabular}
\end{table}

which uses the rates that can be derived from the Absolute Confusion Matrix. Note: the sum of the
first row should be 1 as well as the second row. In this relative Matrix, there are 4 entries
and 2 contsants which implies that there are 2 degrees of freedom. This means that we can 
use 2 variables from the the relative confusion table in order to plot the entire entire table.

\paragraph{ROC as 2D Measure}
Thus, the ROC is a 2D measure thar uses (FPR, TPR).
Once we plot this measure, we can change the threshold we're using $\Theta$ and plot these
different values of $\Theta$.
When one plots all values of $\Theta$, then any point on the curve plotted specifies a relative 
confusion matrix and any relative confusion matrix specifies a point on the ROC curve (for a 
given system)

\subsection*{Example}
Consider the following plots where 5 Relative Confusion Matricies of Classifiers are plotted
(left) and where 1 point is iterated over time to increase the threshold for classification
$\Theta$
\incimg{ROCex}{0.5}

\section*{c ROC Curve of Ficticious Virus}
\begin{enumerate}
        \item variant of a virus infects most people quite promptly
        \item variant infects most people slowly and eventually
\end{enumerate}
\incimg{variants}{0.8}

Say we want to be able to determine if someone has a variant while only testing
for the second variant.
\pagebreak

\paragraph{Solution}
The cumulative detection of the virus is equal to the area under the curve. The number of
days after the onset of the virus is $\Theta$ where the time someone has had the virus determines
which variant they have.

\incimg{AUC}{0.7}

Note: red = variant 1, blue = variant 2\\
Observe that the area under the red curve but above the blue curve are true positives for
variant 1 and the area under the blue \& red curves represents the false positives for variant 1.

\paragraph{Vary $\Theta$}
We shall vary the hyper-parameter $\Theta$ from $0:20$. For small $\Theta$: FPR = 0 and TPR = 0
where FPR is magenta and TPR is cyan.
\incimg{fprtpr}{0.5}

we can now find the ROC Curve where the independent variable $\Theta$ has the dependent
variables FPR and TPR. The area under the curve is usually $0.5 \leq AUC \leq 1$
\incimg{ROCCurve}{0.4}

The AUC is 0.86 which is not very good, this suggests that time is not ideal for 
diagnosing variant.

\subsection*{ROC Summary}
\begin{itemize}
        \item Confusion matirx - absolute and relative methods
        \item Key measures - FPR, TPR
        \item $2 \times 2$ relative confusion matirx has 2 degrees of freedom (can be represented 
                by 2 of the four entries)
        \item ROC Curve - plots hyper-parameter changes
        \item AUC - gives the improvement of the method relative to the null hypothesis.
\end{itemize}

\section*{Learning Outcomes}
Students should now be able to:
\begin{itemize}
        \item Compute a relative confusion matirx
        \item find the ROC point for a confusion matrix
        \item plot the ROC curve for a hyper-parameter $\Theta$
        \item asses the ROC curve using AUC
\end{itemize}

\end{document}

