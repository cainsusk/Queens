\documentclass[12pt]{book} 

\usepackage{amsmath}
\usepackage{graphicx}
\usepackage{import}
\usepackage{amsfonts}
\usepackage{booktabs}

\setlength{\parindent}{0em}  % sets auto indent at new paragraph to none

\newcommand{\incfig}[1]{%
        \import{./figures/}{#1.pdf_tex}
}

\newcommand{\incimg}[2]{%
       \begin{figure}[h]
               \centering
               \includegraphics[scale = #2]{./figures/#1}
       \end{figure}
}

\title{\coursetitle\linebreak\lecturename}
\author{\\Cain Susko\\ 
           \\ \\ \\
      Queen's University 
    \\School of Computing\\} 

%=-=-=-=-=-title-=-=-=-=-=%
\newcommand{\lecturename}{Odds of Occurrence and Probability}
\newcommand{\coursetitle}{Linear Data Analysis}
%=-=-=-=-=-#####-=-=-=-=-=%

\begin{document}
\begin{titlepage}
        \maketitle
\end{titlepage}


\section*{a Odds of Hyperplane Classification}
Previously, We dealt with a hyperplane $\mathbb{H}$ and a vector $\vec u$ such 
that $\vec u$ is on one of 2 sides of the $\mathbb{H}$. This is useful
but there is more information we can use to determine the location of $\vec u$ 
relative to $\mathbb{H}$.
For example, the measurements of $\vec u$ are taken in a coordinate frame. 
Hyperplane $\mathbb{H}$ will have values in the same coordinate frame. This 
allows us to create a reference point on $\mathbb{H}$ called $\vec r$ and
a corresponding unit vector $\vec n$; which allows us to determine the 
probability that $\vec u$ is on one side of $\mathbb{H}$ or the.
This is to account for when points are very
close to the hyperplane as the labeling is not as certain as the binary
classification makes it seem.

\paragraph{Terms}
The definition of terms for probability in this course are:

\begin{align*}
        &\text{Odds} &\text{ratio of likelihood of an event}\\
        &\text{Probability} &\text{inverse function of odds}\\
        &\text{Logistic Function} &\text{maps a score to a probability}\\
\end{align*}

We shall be dealing with the \textit{Odds} that a data point will be in a given 
set.

\section*{b Odds and Probability}
we shall first show by example what we mean by Odds. Given Odds (written as 
$S$), we say that $S$ is the evaluation of the following ratios:


\begin{align*}
       &50:50 &S=1\\
       \\
       &9:1 &S=9\\
       \\
       &1:4 &S=\frac{1}{4}
\end{align*}
\pagebreak

From these examples we can then infer that the Odds from a given ratio are: 
\[S = \frac{p}{1-p}\] 
where we are given $a:b$ such that $p = \frac{a}{a+b}$ which is the probability
of the action occurring.
This implies that $S$ is greater than 0 with no upper bound.

\paragraph{Finding Probability}
Given the Odds $S$, can we find the probability $p$?
\begin{align*}
        S &= \frac{p}{1-p}\\
        &\equiv S(1-p) = p\\
        &\equiv S-Sp = p\\
        &\equiv S = p+Sp\\
        &\equiv S = p(1+S)\\
        &\equiv p = \frac{S}{S+1}
\end{align*}

Thus, the formula for finding the probability of an outcome  given it's odds is:
\[p = \frac{S}{1+S}\]

\section*{c Logistic Function for Odds}
this section relates the score of a data point to it's probability using 
the Logistic Function. The bounds of an Odd and a Score are:
\begin{align*}
        &\text{Odd} &S \in (0, +\infty)\\
        &\text{Score} &z \in (-\infty, +\infty)\\
\end{align*}

The Logistic function should map an Odd to an Score such that:
\[f:S\mapsto z\]
where the resulting graph is invertible, smooth, 1:1, and monotone.
These constrains allow for the score to equal:
\[z = ln(S)\]
such that:
\[S = e^z\]

Now, given a hyperplane $\mathbb{H}$, a unit normal vector $\vec n$ and
reference point $a$ (same as weights vector and bias scalar for $\mathbb{H}$)
. The score for a given value $\vec u$ is equal to:
\[z(\vec u) = \vec n^\top \vec u + a\]

Now, to relate all these inferences, the way we relate a probability to
a score is:
\begin{align*}
        p(z) &= \frac{e^z}{1+e^z}\\
        \\
        &= \frac{1}{1+e^{-z}}
\end{align*}

A consequence of this equation is that if $z$ is negative, then:
\[p(-z) = 1-p(z)\]

Thus the Odds of $\vec u$ of being in the \textbf{positive half-space} of 
$\mathbb{H}$ is:
\[z(\vec u) = \vec n^\top \vec u + a\]
\[p(\vec u; \vec n, a) = \frac{1}{1+e^{-(\vec n^\top \vec u + a)}}\]

\section*{d Properties of The Logistic Function}
given the scores and probability of data-vectors from a logistic function we
can plot the following graph:
\incimg{logFunc}{0.5}

Where the white line is the Logistic function and the green line is the 
binary classification function.
Additionally the x axis is the score and the y axis is the probability 
for $\vec u$.

\subsection*{Derivative of p(z)}
The equation for relating score to a probability is:
\[p(z) = \frac{e^z}{1+e^z}\]
observe:
\begin{align*}
        1-p(z) &= \frac{1+e^z}{1+e^z}-\frac{e^z}{1+e^z}\\
               &= \frac{1}{1+e^z}\\
       \\
               &= p(-z)\\
\end{align*}

\paragraph{Recall}
The derivative of a fraction is:
\[\frac{d}{dz}\left(\frac{f(z)}{g(z)}\right) = \frac{f'g - fg'}{g^2}\]
Additionally, the derivative of $e^x$ is:
\[\frac{d}{dz}e^z = e^z\]

\paragraph{Solution}
The formula for the derivative of $p(z)$ is:
\begin{align*}
        \frac{d}{dz}p(z) &= \frac{e^z(1+e^z)-e^ze^z}{(1+e^z)^2}\\
        \\
        &= p(z)(1-p(z))
\end{align*}

\section*{Summary}
\paragraph{Probability}
\begin{itemize}
        \item Odds: ratio involving probability
        \item Probability: function of distance from hyperplane
\end{itemize}

\paragraph{Logistic Function}
\begin{itemize}
        \item continuous, differentiable, invertible
\end{itemize}

\section*{Learning Outcomes}
Students should now be able to:
\begin{itemize}
        \item Transform between Odds and Probability
        \item Find unit form of hyperplane $\mathbb{H}$
        \item find probability that $\vec u$ is on positive side of $\mathbb{H}$
\end{itemize}

\end{document}

