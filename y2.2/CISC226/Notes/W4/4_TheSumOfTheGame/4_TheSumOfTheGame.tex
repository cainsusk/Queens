\documentclass[12pt]{book} 

\usepackage{amsmath}
\usepackage{graphicx}
\usepackage{import}

\setlength{\parindent}{0em}  % sets auto indent at new paragraph to none


\newcommand{\incfig}[1]{%
    \import{./figures/}{#1.pdf_tex}
}

\title{\coursetitle\linebreak\lecturename}

\author{\\Cain Susko\\ 
           \\ \\ \\
      Queen's University 
    \\School of Computing\\} 

%=-=-=-=-=-title-=-=-=-=-=%
\newcommand{\lecturename}{Considering a Game as the Sum of its Parts}
\newcommand{\coursetitle}{Game Design}
%=-=-=-=-=-#####-=-=-=-=-=%

\begin{document}
\begin{titlepage}
        \maketitle
\end{titlepage}


\section*{Goal}
What is unique about video games is that the player is one of the most integral parts of the game.
Because of this, the goal of the player is very important to the design of the game.
A player takes on a Lusory attitude which means that the player accepts the rules of the game despite of its unintuitive
(soccer, u play with your feet).
\begin{itemize}
        \item Jeproady
        \begin{itemize}
                \item very rigid goals, target is money
        \end{itemize}
        \item Monopoly
        \begin{itemize}
                \item non rigid, a fun game you play with your family to bond
        \end{itemize}
        \item SimCity
        \begin{itemize}
                \item the goal is to make the city to your liking. 
                        There are stats that tell you how you're doing but its player driven
        \end{itemize}
\end{itemize}

\section*{Procedure}
a procedure of a game is a method through which to play the game, as permitted by the rules.
This encompasses procedures mandated by rules, social conventions, and control schemes.

\section*{Rules}
the rules of the game are what objects the game consists of and what the players can and cannot do.
For example, the objects of Chess are the pieces and what they can and cannot do are the rules of the objects.
The rules in video games are the same accept they can tend to be much more complex (trading items, virtual economy, 
large complex systems etc.)

\section*{Resources}
The resources of the game commonly are used to restrict the actions of the player to be more strategic and to make it
        more difficult.
For example, Mana for casting spells etc.

\section*{Conflict}
Conflict of a game is the competitive or driving aspect of the game.
Conflict can be other players, NPCs, the environment, the economy, or even time (time trails in Mariokart).
What is the player's source of conflict?

\section*{Boundaries}
Within games there is a 'space' in which the games are played. (you can punch people in hockey but not in the parking lot after)

\section*{Outcome}
What does the player get from playing the game? What do they accomplish. This may be different than the players Goals for the game.
The outcome may be that they player was defeated by another player or AI.
The uncertainty of the outcome is \textbf{very} important to the playability and enjoyability of the game.
\end{document}

