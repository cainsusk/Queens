\documentclass[12pt]{book} 

\usepackage{amsmath}
\usepackage{graphicx}
\usepackage{import}
\usepackage{amsfonts}
\usepackage{booktabs}

\setlength{\parindent}{0em}  % sets auto indent at new paragraph to none

\newcommand{\incfig}[1]{%
        \import{./figures/}{#1.pdf_tex}
}

\newcommand{\incimg}[2]{%
       \begin{figure}[h]
               \centering
               \includegraphics[scale = #2]{./figures/#1}
       \end{figure}
}

\title{\coursetitle\linebreak\lecturename}
\author{\\Cain Susko\\ 
           \\ \\ \\
      Queen's University 
    \\School of Computing\\} 

%=-=-=-=-=-title-=-=-=-=-=%
\newcommand{\lecturename}{Assignment 4}
\newcommand{\coursetitle}{Software Specifications}
%=-=-=-=-=-#####-=-=-=-=-=%

\begin{document}
\begin{titlepage}
        \maketitle
\end{titlepage}


\begin{enumerate}
        \item The Precondition for the given statements should be:
        \begin{enumerate}
                \item \texttt{x == 1}
                \item \texttt{x == 2}
                \item \texttt{0 < 2y + 2z}
                \item \texttt{2(y*z + 3) > y + 2}
                \item \texttt{Exists(y = 0; y < 10) 2y + 1 == 50}
                \item \texttt{Exists(y = 0; y < 15) 2(x+y) == y + t}
                \item \texttt{ForAll(z =1; z < 100) 3y + 1 > z + 2}
                \item \texttt{ForAll(y = 1; y < x) 2x + 2y < 100}
                \item \texttt{Exists(y+z = 0; y+z < 10) z*z + 2(y + z)== 15}
                \item \texttt{Exists(y = 0; y < 100)(2y + z == 15 || 2z*y + y < 100)}
        \end{enumerate}
        \item These following statements have been verified:
        \begin{enumerate}
                \item \begin{verbatim}
ASSERT( x >= 5 || ( x <= 0 && y == 2 ) )
z = x - y;
ASSERT(y == 2)
y = y + z;
ASSERT(y - z == 2 || y > 3)
x = y - z;
ASSERT( x == 2 || y > 3 )
                \end{verbatim}
                \item \begin{verbatim}
ASSERT( z >= 0 )
if ( x < 2 ) { y = z+3; } //end-if 
ASSERT(y > x + 1 || x + z > 1)
z = x + z;
ASSERT(y > x + 1|| z > 1)
                \end{verbatim}
        \end{enumerate}

        \item \begin{enumerate}
                \item The Loop Invariant is:
                        \texttt{i > 0}
\pagebreak
                \item The Complete Proof Tableau is:
                        \begin{verbatim}
ASSERT(k >= 0)
i = k;
ASSERT(i > 0)
sum = k;
while( i > 0 )
{
        ASSERT(sum+i-1 == j && i > 0)
        i = i - 1;
        ASSERT(sum+i == j && i > 0)
        sum =sum+i; 
}
ASSERT(sum == SUM{j=0->k}(j))
                        \end{verbatim}
        \end{enumerate}

        \item \begin{verbatim}
const int n; /* the program will 
                compute the sum of the squares
                of the n smallest positive odd integers*/
int sum;     /* the sum is stored in this variable */

int count = 0;
int i = 0;

ASSERT( n >= 1 )

while(count < n)
        i++
        ASSERT(count < n)
        if (i % 2 != 0){
                ASSERT(count < n && i*i == 
                        (2*count)*(2*count))
                count++;
                ASSERT(count < n && i*i == 
                        (2*count+1)*(2*count+1))
                sum += i*i;
        }
}
ASSERT( sum == SUM{i=0->n-1} (2*i+1)*(2*i+1) )
        \end{verbatim}

        The loop invariant is \texttt{count < n} and the
        agrgument for any sequene of numbers there are a given 
        number of odd numbers. Becuase the program can generate a functionally endless sequence of numbers, there will always be to one
        point $n$ odd number is a series. this is not always 
        possible for extremely large values of $n$
\end{enumerate}

\end{document}

