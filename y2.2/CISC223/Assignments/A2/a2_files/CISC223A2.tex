\documentclass[12pt]{book} 

\usepackage{amsmath}
\usepackage{graphicx}
\usepackage{import}
\usepackage{amsfonts}
\usepackage{booktabs}
\usepackage{enumitem}

\setlength{\parindent}{0em}  % sets auto indent at new paragraph to none

\newcommand{\incfig}[1]{%
    \import{./figures/}{#1.pdf_tex}
}

\title{\coursetitle\linebreak\lecturename}
\author{\\Cain Susko\\ 
           \\ \\ \\
      Queen's University 
    \\School of Computing\\} 

%=-=-=-=-=-title-=-=-=-=-=%
\newcommand{\lecturename}{Assignment 2}
\newcommand{\coursetitle}{Software Specifications}
%=-=-=-=-=-#####-=-=-=-=-=%

\begin{document}
\begin{titlepage}
        \maketitle
\end{titlepage}


\section*{Question 1}
(4 marks) Using the method described in Section 9.1 and in videos of Week 3 (video 12)
convert the following regular expression into a state diagram:
\[
        (01\text{*}+10)\text{*}1\text{*}
.\] 
\subsection*{Answer}
To convert a Regex to a NFA we must first create a state diagram for the most basic parts of this regex:
        $0,1$.
 \begin{figure}[h]
         \centering
         \incfig{Q1pt1}
         \caption{0 and 1}
\end{figure}

Following the steps from Lesson 12, we will now find the concatenation of 1 and 0:

\begin{figure}[h]
        \centering
        \incfig{Q1pt2}
         \caption{10}
\end{figure}
\pagebreak

Next, we shall create a state diagram for the other half of the union:
\begin{figure}[h]
        \centering
        \incfig{Q1pt3}
        \caption{1*}
\end{figure}
\begin{figure}[h]
        \centering
        \incfig{Q1pt4}
        \caption{01*}
\end{figure}
\pagebreak


Now that we have each respective part of the regex in the parenthesis, we can now represent the union:
\begin{figure}[h]
        \centering
        \incfig{Q1pt5}
        \caption{01*+10}
\end{figure}

And next we will apply closure to the above diagram:
\begin{figure}[h]
        \centering
        \incfig{Q1pt6}
        \caption{(01*+10)*}
\end{figure}
\pagebreak

Finally, we will concatenate the above diagram with 1*:
\begin{figure}[h]
        \centering
        \incfig{Q1pt7}
        \caption{(01*+10)*1*}
\end{figure}

And thus, this is a corresponding state diagram for the given Regex.

\section*{Question 2}
(3 marks) Using the method described in Section 9.2 (and in videos of Week 3), convert
the state diagram given in Figure 1 into an equivalent regular expression. Where $\Sigma$ =
{a, b, c, d}.

\subsection*{Answer}
{\small
From looking at the given state diagram, one can see that there is only one possible node to remove
        while using the state elimination algorithm as one cannot remove the starting or final state.
Thus, my intermediate intermediate state diagram is:}
\begin{figure}[h]
        \centering
        \incfig{Q2pt1}
\end{figure}

Thus, we can now derive the Regex from the above diagram:
\[
        (bd\text{*}c)\text{*}(bd\text{*}a)
        [c+bd\text{*}a+(a+bd\text{*}c)(bd\text{*}c)(bd\text{*}a)]\text{*}
.\] 

Therefore, we have found a corresponding regex for the given state diagram.

\section*{Question 3}
(4 marks)Are the following languages A and B over the alphabet $\Sigma = {a, b, c, d}$ regular or non-regular?
\begin{enumerate}[label=\alph*)]
        \item $A = \{a^{2i} b^{3k+1} c^{m+3} \mid i,k,m \geq 0\}\cup\{b^{3r+1} c^{2s+4} d^{5t+1} \mid r,s,t \geq 0\}$
        \item $B = \{d^{2i} c^{3i} b^{2k+1} \mid i,k\geq 1\}\cdot\{b^{3m+2} a^{2r+1} \mid m,r\geq 1\}$
\end{enumerate}
\subsection*{Answer}
\begin{enumerate}[label=\alph*)]
        \item We will first attempt to make a Regex for the given set $A$ by creating a  $\epsilon$-NFA and converting it to an
                NFA then a Regex

                From this operation we can derive the following Regex:
                 \[
                [(aa)\text{*}(bbb)\text{*}bc\text{*}ccc] + [(bbb)\text{*}b(cc)\text{*}cccc(ddddd)\text{*}d]
                .\] 

                Thus, this Regex represents $A$. Therefore,  $A$ is \textbf{regular}
                
        \item From initial observation we can infer that $B$ may be irregular, thus we will first try to show it is irregular using the
                pumping lemma and proof by contradiction.

                For the sake of contradiction we will assume $B$ is regular. Let  $n$ be the constant given by the pumping lemma. 
                Let $x=d^{2n} c^{3n} b^{2n+1} b^{3n+2} a^{2n+1}$.
                By the pumping lemma, any string $x=pqr$ which satisfies the pumping lemma must be in  $B$.
               Let $q\in x = c^{3j}\;\forall_j\{j\geq 0\}$ 
                \[
                x = pq^2r = d^{2n} c^{3n+3j} b^{2n+1} b^{3n+2} a^{2n+1} \not\in B
                .\] 
                
                Thus, we have reached a contradiction as for $B$ to be regular is must contain the above string; which it does not.
                Therefore,  $B$ is \textbf{irregular}
\end{enumerate}
\pagebreak


\section*{Question 4}
{\small(3 marks) Using the algorithm mark distinguishable pairs of states that was presented in
videos of Week 4 (videos 23–24) and that can be found in the course notes, minimize the
number of states of the DFA depicted.}

\subsection*{Answer}
We must minimize the given State Diagram in 2 steps:
\begin{enumerate}
        \item[\textbf{Stage 0}] First, we will mark all pairs of states that are each final and non-final with a 0:
                \begin{table}[h]
                \centering
                \begin{tabular}{@{}llllll@{}}
                \toprule
                  & A & B & C & D & E \\ \midrule
                A & X &   &   &   &   \\
                B &   & X &   &   &   \\
                C &   &   & X &   &   \\
                D &   &   &   & X &   \\
                E &   &   & 0 & 0 & X \\ \bottomrule
                \end{tabular}
                \end{table}
        \item[\textbf{Stage 1}] Next, we will mark any pairs of states that transition--over the same symbol--to a previously 
                marked pair of states. We will mark \textit{these} states with 1.
                \begin{align*}
                        &(A,D)\to^b (C,E)
                        &(A,B)\to^b (C,E)\\
                        &(C,D)\to^a (A,B)
                        &(A,C)\to^a (B,A)\\
                        &(B,C)\to^a (D,A)
                        &(A,E)\to^a (B,C)\\
                        &(B,E)\to^a (C,D)
                .\end{align*}
                \pagebreak
                \begin{table}[h]
                \centering
                \begin{tabular}{@{}llllll@{}}
                \toprule
                  & A & B & C & D & E \\ \midrule
                A & X &   &   &   &   \\
                B & 1 & X &   &   &   \\
                C & 1 & 1 & X &   &   \\
                D & 1 &   & 1 & X &   \\
                E & 1 & 1 & 0 & 0 & X \\ \bottomrule
                \end{tabular}
                \end{table}
                
                Thus, states D and B are indistinguishable and the resulting minimized State Diagram is:
                \begin{figure}[h]
                        \centering
                        \incfig{Q4}
                \end{figure}

\end{enumerate}

\section*{Question 5}
(6 marks) Consider languages over terminal alphabet $\Sigma = \{a, b, c, d\}$.


• Give context-free grammars that generate the following four languages.


• In each case, also give a derivation of the specified terminal string using your grammar. 
\begin{enumerate}[label=\alph*)]
        \item $A = \{a^ib^{2m} c^{3i} \mid i,m\geq 1\}\cup\{b^jc^{3k}d^k \mid j,k\geq 1\}$\\
                Derivation for the string $bc^9d^3$
        \item $B = \{a^{3i} b^{4k} c^{3k}d^{2i}  \mid i,k\geq 0\}$ Derivation for  $a^3b^8c^6d^{2}$
        \item $C = \{a^{3i}b^{4i}c^{3k} d^{2k+1} \mid i,k\geq 0\}$ Derivation for $c^3 b^4 c^9 d^7$
        \item $D = \{a^{3i+1}b^{i+3}c^{2k} d^{2k} \mid i,k\geq 0\}\cup \{a^{2r}b^sc^{5s}d^{r+1} \mid r,s\geq 0\}$\\
                Derivation for the string $a^2bc^5d^2$
\end{enumerate}

The derivation beginning from the start variable should indicate each individual
derivation step using the notation $\Rightarrow$.
\subsection*{Answer}
\begin{enumerate}[label=\alph*)]
        \item Grammar: $\{a^ib^{2m} c^{3i} \mid i,m\geq 1\}\cup\{b^jc^{3k}d^k \mid j,k\geq 1\}$
                \begin{align*}
                        S &\to N \mid W\\
                        N &\to aNc^3  \mid aOc^3\\
                        O &\to b^2O  \mid b^2\\
                        W &\to bW \mid bY\\
                        Y &\to c^3Yd  \mid c^3d
                .\end{align*}
                \[
                S\implies W \implies bY \implies bc^3Yd \implies bc^6Yd^2 \implies bc^9d^3
                .\]

        \item Grammar: $\{a^{3i} b^{4k} c^{3k}d^{2i}  \mid i,k\geq 0\}$
                \begin{align*}
                        B &\to a^3Bd^2 \mid a^3Xd^2 \mid X\\
                        X &\to b^4Xc^3  \mid b^4c^3  \mid \epsilon\\
                .\end{align*}
                \[
                B \implies a^3Xd^2 \implies a^3b^4Xc^3d^2 \implies a^3b^8c^6d^2
                .\] 

        \item Grammar: $\{a^{3i}b^{4i}c^{3k} d^{2k+1} \mid i,k\geq 0\}$
                \begin{align*}
                        C &\to OW\\
                        O &\to a^3Ob^4  \mid a^3b^4 \mid \epsilon\\
                        W &\to c^3Wd^2 \mid c^3d^2d \mid d
                .\end{align*}
                \[
                C \implies OW \implies a^3b^4W \implies a^3b^4c^3Wd^2 \implies a^3b^4c^6Wd^4 \implies a^3b^4c^9d^7
                .\] 

        \item Grammar: $\{a^{3i+1}b^{i+3}c^{2k} d^{2k} \mid i,k\geq 0\}\cup \{a^{2r}b^sc^{5s}d^{r+1} \mid r,s\geq 0\}$
                \begin{align*}
                        D &\to P \mid X\\
                        P &\to QR\\
                        Q &\to a^3Qb \mid a^3abb^3 \mid ab^3\\
                        R &\to c^2Rd^2 \mid c^2dd \mid \epsilon\\
                        X &\to a^2Xd \mid a^2Ydd \mid Yd\\
                        Y &\to bYc^5 \mid bc^5 \mid \epsilon
                .\end{align*}
                \[
                D \implies X \implies a^2Yd^2 \implies a^2bc^5d^2 
                .\] 
\end{enumerate}
\end{document}



