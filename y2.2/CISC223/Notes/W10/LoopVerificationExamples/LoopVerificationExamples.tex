\documentclass[12pt]{book} 

\usepackage{amsmath}
\usepackage{graphicx}
\usepackage{import}
\usepackage{amsfonts}
\usepackage{booktabs}

\setlength{\parindent}{0em}  % sets auto indent at new paragraph to none

\newcommand{\incfig}[1]{%
        \import{./figures/}{#1.pdf_tex}
}

\newcommand{\incimg}[2]{%
       \begin{figure}[h]
               \centering
               \includegraphics[scale = #2]{./figures/#1}
       \end{figure}
}

\title{\coursetitle\linebreak\lecturename}
\author{\\Cain Susko\\ 
           \\ \\ \\
      Queen's University 
    \\School of Computing\\} 

%=-=-=-=-=-title-=-=-=-=-=%
\newcommand{\lecturename}{Loop Verification Examples}
\newcommand{\coursetitle}{Software Specifications}
%=-=-=-=-=-#####-=-=-=-=-=%

\begin{document}
\begin{titlepage}
        \maketitle
\end{titlepage}


\section*{Example}
\begin{verbatim}
       x, y, n, int

       ASSERT(n >= 0)
       i = 0; y = 1;

       while (i != n){
                y = y * x;
                i++;
       } // end while
       ASSERT(y == power(x,n))
\end{verbatim}

to verify this assertion we must find the invariant within the loop. that is, a value that does not change within the loop
and gives the post condition after the loop. the value that satisfies these conditions is:
\[\texttt{y == power(x, i) \&\& 0 <= i <= n}\]

where \texttt{y} is the value and its bounds are specified to the right.

\section*{Example}
consider the following program:
\begin{verbatim}
       y, k, n int;

       ASSERT(n <= 0)
       k = n; y = 1;

       while (k > 0){
                y = y * x;
                k = k - 1;
        } // end while
        ASSERT(y == power(x, n))
\end{verbatim}

thus the invariant is:
\[\texttt{0 <= k <= n \&\& FINISH THIS WITH SLIDE}\]

\section*{Proof Tableau}
this is the proof tableau for the above program:
\begin{verbatim}
       y, k, n int;

       ASSERT(n <= 0)
       ASSERT(0 <= n <= n && 1 == power(x, 0))
       k = n; 
       ASSERT(0 <= k <= n && 1 == power(x, n-k)
       y = 1;
       ASSERT(I)

       while (k > 0){ ASSERT(I && k > 0)
       // k > 0 implies 0 <= k-1 when k is int
       // y == power(x, n-k) implies y * x == power(x, n-k+1)
                ASSERT(0 <= k-1 <= n && y * x == power(x, n-k+1)
                y = y * x;
                ASSERT(0 <= k-1 <= n && y * x == power(x, n-k+1)
                k = k - 1;
                ASSERT(I)
        } // end while
        ASSERT(I && k <= 0)
        // k <= 0 && 0 <= k implies k == 0, k == 0 && I implies post-condition 
        ASSERT(y == power(x, n))
\end{verbatim}

Note, the invariant \texttt{I} is \texttt{FUCK YOU WITH THE GODDAMN PAPER GET SOME FUNCING SLIDES LIKE WERE IN THE 21st CENTURY}

Termination: according to the invariant $k \leq n$ and each iteration of the loop should FINISH WITH SLIDES

\section*{Program Termination}
checking wether a while loop terminates is a n algorithmically unsolvable problem. At the end of the this course we
will discull unsolvability.

for a "well structured" while loop we can argue termination by including suitable bounds for the loop variable in
the invariant.

\section*{Algorithm for Verification}
Finally, the algorith for verifying the proof tableau of a while loop is as follows:
\begin{enumerate}
        \item[i] Select an invariant and show that the loop body preserves the invariant.
        \item[ii] Show that the invariant holds before entering into the loop
        \item[iii] Show that invariant and negation of loop condition implies the post contidion--possibly
                after some termiation assignments.
        \item[iv] Argue that the loop terminates--typically done by including the bounds for the loop
                variable in the invariant
\end{enumerate}

\section*{Guidelines for Invariant}
\begin{itemize}
        \item often the invariant can be an generalization of the post-condition -- replace some constant in the post-condition by 
                one in the loop variable.
        \item normally we add bounds for the loop variable in the invariant (for arguing termination)
\end{itemize}

\section*{More Example}
consider the following:
\begin{verbatim}
        ASSERT(0 <= n <= max)
        { int i;
        present = false;
        i = 0;

        while (i != n){
                if(A[i] == x) present = true; i++;
        } // end while
        ASSERT(present iff x in A[0:n-1])
\end{verbatim}

Thus, the invariant is just a generalization of the post-condition:
\[\texttt{present iff x in A[0:i-1] \&\& 0 <= i <= n}\]

where the part on the right are the bounds for termination.


\end{document}

