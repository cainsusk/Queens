\documentclass[12pt]{book} 

\usepackage{amsmath}
\usepackage{graphicx}
\usepackage{import}
\usepackage{amsfonts}
\usepackage{booktabs}

\setlength{\parindent}{0em}  % sets auto indent at new paragraph to none

\newcommand{\incfig}[1]{%
        \import{./figures/}{#1.pdf_tex}
}

\newcommand{\incimg}[2]{%
       \begin{figure}[h]
               \centering
               \includegraphics[scale = #2]{./figures/#1}
       \end{figure}
}

\title{\coursetitle\linebreak\lecturename}
\author{\\Cain Susko\\ 
           \\ \\ \\
      Queen's University 
    \\School of Computing\\} 

%=-=-=-=-=-title-=-=-=-=-=%
\newcommand{\lecturename}{Verifying If Statement}
\newcommand{\coursetitle}{Software Specifications}
%=-=-=-=-=-#####-=-=-=-=-=%

\begin{document}
\begin{titlepage}
        \maketitle
\end{titlepage}


\section*{Verifying If Statements}
\subsection*{If Part}
\begin{verbatim}
ASSERT(I && i != n && A[i] != B[i])
// i <= n && i != n implies i + 1 <= n when i is int
// A[i] != B[i] implies that 
ASSERT(0 <= i+1 <= n && false iff ForAll(k=0, k<i+1)A[b] = -B[k])
        equal = false
ASSERT(0 <= i + 1 < = n && eual iff FofAll(k=0; k < i+1)A[k]==B[k])
\end{verbatim}

\subsection*{Else Part}
\begin{verbatim}
I && i != n && A[i] == B[i] implies (*) 0 <= i+1 <= n && 
        equal iff ForAll(k=0, k<i+1) A[k] == B[k]

(*) i <= n && i != n implies i+1 <= n when i,n are ints 
        (equal iff ForAll(k=0, k<i) A[k] == B[k]) && A[i] == B[i] 
        implies equal iff ForAll(k = 1, n < i+1)A[k] == B[k]
\end{verbatim}

\section*{Verifying For Loops}
For Loops are written in C like so: \texttt{for(i=n; i<N; i++)C}
. The general form of a for loop is represented as:
\[for(A_o; B; A_1)\]

A verification works like so:
\begin{verbatim}
ASSERT(P) // pre-condition
        ...
A_o;
ASSERT(I) // I invariant
while(B){
        ASSERT(I&&B)
                CA_1;
        }
ASSERT(I&&!B)
\end{verbatim}
Consider the following example:
\begin{verbatim}
ASSERT(0 <= n < max){
        int i;
        for (i = 0; A[i] != x &&i < n; i++)
                {}
        present = (i<n)
ASSERT(present iff x in A[0:n-1])
\end{verbatim}
to verify it we must frist re write it as a while loop:
\begin{verbatim}
ASSERT(0 <= n < max){
        int i;
        for (i = 0; A[i] != x &&i < n; i++)
                {i++;}
        ASSERT(i < n iff x in A[0:n-1]) // while loop post condition
        present = (i<n)
ASSERT(present iff x in A[0:n-1])
\end{verbatim}
The Invariant \texttt{I} is bounded by \texttt{0 <= i <= n} from the while loop 
bounds but what is the invariant it's self. The idea is we continue the 
execution of the loop untill we find the target element, thus the first \texttt{i}
entiries are not equal to \texttt{x}, therefore the invariant is:
\[0 \geq i \geq n \wedge \{\forall_k k > i\}\;\;x \neq A[k]\]

\end{document}

