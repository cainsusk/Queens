\documentclass[12pt]{book} 

\usepackage{amsmath}
\usepackage{graphicx}
\usepackage{import}
\usepackage{amsfonts}
\usepackage{booktabs}

\setlength{\parindent}{0em}  % sets auto indent at new paragraph to none

\newcommand{\incfig}[1]{%
        \import{./figures/}{#1.pdf_tex}
}

\newcommand{\incimg}[2]{%
       \begin{figure}[h]
               \centering
               \includegraphics[scale = #2]{./figures/#1}
       \end{figure}
}

\title{\coursetitle\linebreak\lecturename}
\author{\\Cain Susko\\ 
           \\ \\ \\
      Queen's University 
    \\School of Computing\\} 

%=-=-=-=-=-title-=-=-=-=-=%
\newcommand{\lecturename}{Predictive Recursive Descent Parsing Example}
\newcommand{\coursetitle}{Software Specifications}
%=-=-=-=-=-#####-=-=-=-=-=%

\begin{document}
\begin{titlepage}
        \maketitle
\end{titlepage}


\section*{Recall}
given the set of balanced strings, the grammar for this set is:
\[<balanced> \to 0<balanced>1 \mid \epsilon\]

thus, the parser operations for the set are:
\begin{itemize}
        \item if next token is 0, use $<balanced> \to 0<balanced>1$
        \item if next token is 1 or EOS, then use 
                $<balanced> \to \epsilon$
\end{itemize}

This is the process commonly used in program compilers.

\paragraph{Limitations}
there are, however, limitations to what languages can use Recursive 
Descent Parsing. 

take for example, the language of palendromes 
$L_{pal} = \{w\in\{0,1\}\mid w = w^r\}$ where if $w = 110$ then
$w^r = 011$

the Grammar would thus be:
\[S \to 0S0 \mid 1S1 \mid 0 \mid 1 \mid \epsilon\]

which outlines the problems of predictive parsing with $L_{pal}$:
\begin{itemize}
        \item two productions for same variable begin with the same token
                \[S \to 0S0 \mid 0\]
        \item the variable $S$ has productions that begin with 0 \textit{
                and} 0 can occur directly after $S$ \textbf{and} S has
                a erasing production:
                \[S \to 0S0 \;\;\; S \to \epsilon\]
                the problem is there is no way for the parser to know
                which production to use and there is more than 1 option.
\end{itemize}

\end{document}

