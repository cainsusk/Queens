\documentclass[12pt]{book} 

\usepackage{amsmath}
\usepackage{graphicx}
\usepackage{import}
\usepackage{amsfonts}
\usepackage{booktabs}

\setlength{\parindent}{0em}  % sets auto indent at new paragraph to none

\newcommand{\incfig}[1]{%
        \import{./figures/}{#1.pdf_tex}
}

\newcommand{\incimg}[2]{%
       \begin{figure}[h]
               \centering
               \includegraphics[scale = #2]{./figures/#1}
       \end{figure}
}

\title{\coursetitle\linebreak\lecturename}
\author{\\Cain Susko\\ 
           \\ \\ \\
      Queen's University 
    \\School of Computing\\} 

%=-=-=-=-=-title-=-=-=-=-=%
\newcommand{\lecturename}{Recursive Descent Parsing}
\newcommand{\coursetitle}{Software Specifications}
%=-=-=-=-=-#####-=-=-=-=-=%

\begin{document}
\begin{titlepage}
        \maketitle
\end{titlepage}


\section*{Parsing}
Parsing is the process of determining whether a string of tokens can be
generated by a grammar. There are many types of parsing:
\begin{itemize}
        \item Brute Force Parsing--systematically tries all possible 
                derivations. $O(2^n)$
        \item dynamic programming algorithm--General CF Grammars can
                be parsed in $O(n^3)$ 
        \item sophiticated Parsing Table--can parse a Deterministic CF 
                Grammar in $O(n)$
\end{itemize}

\paragraph{Recursive Paring}
we shall now consider a Deterministic CF Grammar using Predictive
Recursive Descent parsing. to do this we must:
\begin{itemize}
        \item associate a procedure to each non-terminal of the grammar
        \item the parser then decides which production to use for a given 
                non-terminal based only on the current input token
\end{itemize}

\section*{Example}
Given the set of balanced strings: $\{0^i1^i \mid i \geq 0\}$
it would have the followiung grammar:
\begin{align*}
        <balanced> \to 0<balanced>1 \mid \epsilon
\end{align*}
and thus, a possible predictive parser for this grammar is:
\begin{itemize}
        \item if next token is 0, use $<balanced> \to 0<balanced>1$
        \item if next token is 1, use $<balanced> \to \epsilon$
\end{itemize}
\end{document}

