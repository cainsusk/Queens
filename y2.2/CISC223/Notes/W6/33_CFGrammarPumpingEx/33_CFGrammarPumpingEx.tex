\documentclass[12pt]{book} 

\usepackage{amsmath}
\usepackage{graphicx}
\usepackage{import}
\usepackage{amsfonts}
\usepackage{booktabs}

\setlength{\parindent}{0em}  % sets auto indent at new paragraph to none

\newcommand{\incfig}[1]{%
    \import{./figures/}{#1.pdf_tex}
}

\title{\coursetitle\linebreak\lecturename}
\author{\\Cain Susko\\ 
           \\ \\ \\
      Queen's University 
    \\School of Computing\\} 

%=-=-=-=-=-title-=-=-=-=-=%
\newcommand{\lecturename}{Context Free Grammar Pumping Lemma Examples}
\newcommand{\coursetitle}{Software Specifications}
%=-=-=-=-=-#####-=-=-=-=-=%

\begin{document}
\begin{titlepage}
        \maketitle
\end{titlepage}


\section*{Example}
Consider the Language of Squares:
\[
L_2 = \{ww \mid w\in\{a,b\}*\}
.\] 
The claim is that $L_2$ is not context free.

Note: showing that $L_2$ is not CF, roughly speaking, shows that variable declarations
cannot be specified with CF Grammars.
\paragraph{attempt 1}
The Question is what string $s$ should we use to derive a contradiction with the 
Pumping Lemma? The first (bad) idea for $s$ is the following:
 \[
s = a^pba^pb\in L_2\\
.\] 
\[
s = a^{p-1}aba^{p-1}b
.\] 
Where:
\begin{align*}
        u &= a^{p-1} \\
        v &= a \\
        w &= b \\
        x &= a \\
        y &= a^{p-1}b \\
.\end{align*}
Let p be the constant yielded from the pumping lemma.
Additionally as given by the pumping lemma, $s = uvwxy$ for all context 
free languages.

However, there is \textbf{no contradiction} with this $s$ as  $v$ and
$x$ can be repeated in parallel.

\paragraph{attempt 2}
We have to show that \textbf{any} way of writing a string in 5 parts, as per the 
pumping lemma, does \textit{not} satisfy the pumping lemma in order to show 
that a language is context free. We need to also make sure that the middle part
of the string ($vwx$) with length at most $p$.

Thus, we will try a better idea for $s$:
 \[
s = a^pb^pa^pb^p\\
.\] 
\pagebreak

Thus, we will start our proof:

For the sake of contradiction assume that $L_2$ is context free and let $p$ be the 
constant given by the pumping lemma

MUST COMPLETE NOTE
\[
s = a^pb^pa^pb^p
.\] 

\end{document}

