\documentclass[12pt]{book} 

\usepackage{amsmath}
\usepackage{graphicx}
\usepackage{import}
\usepackage{amsfonts}
\usepackage{booktabs}

\setlength{\parindent}{0em}  % sets auto indent at new paragraph to none

\newcommand{\incfig}[1]{%
        \import{./figures/}{#1.pdf_tex}
}

\newcommand{\incimg}[2]{%
       \begin{figure}[h]
               \centering
               \includegraphics[scale = #2]{./figures/#1}
       \end{figure}
}

\title{\coursetitle\linebreak\lecturename}
\author{\\Cain Susko\\ 
           \\ \\ \\
      Queen's University 
    \\School of Computing\\} 

%=-=-=-=-=-title-=-=-=-=-=%
\newcommand{\lecturename}{Grammars and Recursive Descent}
\newcommand{\coursetitle}{Software Specifications}
%=-=-=-=-=-#####-=-=-=-=-=%

\begin{document}
\begin{titlepage}
        \maketitle
\end{titlepage}


\section*{Recursive Descent Parsing}
A context free grammar allows recursive descent parsing iff for any two productions having the
same variable on the left side:
\[N \to \alpha \mid \beta\]
the follwowing conditon holds:
\begin{itemize}
        \item[RD1] $FIRST(\alpha)\cap FIRST(\beta)=\emptyset$
        \item[RD2] if $\beta \Rightarrow\text{*} \epsilon$ then:\\
                $FIRST(\alpha)\cap FOLLOW(N) = \emptyset$
\end{itemize}

\section*{Example}
given:
\begin{align*}
        & S\to cAa \mid aAb \mid bB\\
        & A \to dAb \mid cB \mid \epsilon\\
        & B \to bB \mid cBa \mid \epsilon
\end{align*}
to show that Recursive Descent is possible on
the above grammar we must show that it satisfies 
the above requirements, observe the following. Note, we can use the derivations of the above
grammar in order to help find the $FOLLOW$ sets.
\begin{itemize}
        \item[RD1] productions for each variable begin with 
                distinct terminals or are $\epsilon$. thus the grammar holds for RD1
        \item[RD2] $FOLLOW(S) = \{EOS\}$\\
                $FOLLOW(A) = \{a,b\}$\\
                $FOLLOW(B) = \{EOS, a, b\}$\\
                thus, this grammar does not satisfy RD2 as the production $B\to bB$ which is
                makes the following
                \[FIRST(bB)\cap FOLLOW(B)\neq 0\]
                therefore, RD2 does not hold.


        
\end{itemize}
\end{document}

