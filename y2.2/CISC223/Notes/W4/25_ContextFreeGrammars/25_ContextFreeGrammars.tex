\documentclass[12pt]{book} 

\usepackage{amsmath}
\usepackage{graphicx}
\usepackage{import}

\setlength{\parindent}{0em}  % sets auto indent at new paragraph to none

\newcommand{\incfig}[1]{%
    \import{./figures/}{#1.pdf_tex}
}

\title{\coursetitle\linebreak\lecturename}
\author{\\Cain Susko\\ 
           \\ \\ \\
      Queen's University 
    \\School of Computing\\} 

%=-=-=-=-=-title-=-=-=-=-=%
\newcommand{\lecturename}{Context-Free Grammars}
\newcommand{\coursetitle}{Software Specifications}
%=-=-=-=-=-#####-=-=-=-=-=%

\begin{document}
\begin{titlepage}
        \maketitle
\end{titlepage}


\section*{CF Grammar}
A context free grammar is a set of rules for rewriting a language, applied recursively.
A CF grammar can generate many nonregular languages and is commonly used to specify the syntax of a programming language.

\section*{Example 1}
Given the set of balanced strings:
\[A=\{a^i b^i  \mid i \geq 0\}\]
Find a Grammar for $A$
\begin{align*}
        &S\rightarrow aSb\\
        &S\rightarrow\epsilon
.\end{align*}

Where $S$ is the start variable and all expressions to the right of  $S$ are recursive rules for the given set.
The  $S$ in the first rule represents a call of this rule again, hence the recursive property.\\
Combining these rules, we derive the grammar for  $A$:
\[S \Rightarrow aSb\]
which operates like so:
\[aSb\Rightarrow aaSbb\Rightarrow aaaSbbb \Rightarrow \ldots\]

\section*{Example 2}
Given the Expression grammar:
\begin{align*}
        \diamond &\to \diamond +\diamond \\
        \diamond &\to \diamond \times \diamond \\
        \diamond &\to (\diamond)\\
        \diamond &\to a
.\end{align*}
Where $\diamond =\text{expression}$.
 $\diamond $ is \textbf{non-Terminal/Variable}.
Note: the Terminal symbols are $\{+,\times, (, ), a\}$
Thus the derivation of these rules is:
\[\diamond \Rightarrow \diamond + \diamond \Rightarrow \diamond + \diamond \times \diamond \Rightarrow a + \diamond \times \diamond \Rightarrow a+a\times \diamond \Rightarrow a+a\times a  \]
Thus, the terminal string is 
\[a+a\times a\]

From this derivation and terminal string we can create a Parse Tree:
\begin{figure}[h]
        \centering
        \incfig{pt1}
\end{figure}

Note: Grammar is Ambiguous such that some terminal strings can have more than one parse tree. 
For example:
\begin{figure}
        \centering
        \incfig{pt2}
\end{figure}
\end{document}

