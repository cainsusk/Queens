\documentclass[12pt]{book} 

\usepackage{amsmath}
\usepackage{graphicx}
\usepackage{import}

\setlength{\parindent}{0em}  % sets auto indent at new paragraph to none

\newcommand{\incfig}[1]{%
    \import{./figures/}{#1.pdf_tex}
}

\title{\coursetitle\linebreak\lecturename}
\author{\\Cain Susko\\ 
           \\ \\ \\
      Queen's University 
    \\School of Computing\\} 

%=-=-=-=-=-title-=-=-=-=-=%
\newcommand{\lecturename}{Context-Free Grammar Definition}
\newcommand{\coursetitle}{Software Specifications}
%=-=-=-=-=-#####-=-=-=-=-=%

\begin{document}
\begin{titlepage}
        \maketitle
\end{titlepage}


\section*{Definition}
A context free grammar is a tuple $G$ such that
\[
G=\left( V, \Sigma, S, P \right) 
.\] 

Where:
\begin{description}
        \item[$V$] is a finite set of variables or non-terminals
        \item [$\Sigma$] is a finite set of terminals, $V\cap\Sigma\neq\emptyset$
        \item[$S\in V$] is the starting variable
        \item[$P$] is a finite set of productions of the form  $N\to w$ where  $N\in V$ and  $w\in \left( \Sigma\cup V \right)\text{*} $
\end{description}

\section*{Derivation Step}
Consider two strings such that:
\[w_1, w_2 \in (\Sigma\cup V)\]
String $w_1$ derives $w_2$ in one step,
\[w_1\implies w_2\]
if we are able to write:
\[w_1=uNv\wedge w_2=uwv\]
such that $N\to w\in P$;   $n,v,w \in (\Sigma\cup V\text{*})$;   $N\in V$ 
\paragraph{}
Furthermore, the Operator for reflexive trasitive closure on $\Rightarrow$ is denoted by  $\Rightarrow^*$ which can be used if $w_1=w_2$ 
      or if $w_1\implies n_1\implies\ldots\implies n_k\implies w_2$\\
Additionally, a language generated by Grammar $G$ is:
 \[
L(G) = \{w\in\Sigma  \mid S\Rightarrow^*w\}
.\] 
\pagebreak

\section*{Example}
Given:
\[
        A = \{a^{3i}b^k c^{2i+3}  \mid i \geq i, l\geq 1\}
.\] 

The grammar for $A$:
 \begin{align*}
         S&\to a^3Sc^2\\
         S&\to a^3Xc^5\\
         X&\to bX\\
         X&\to b
.\end{align*}

We can then derive the terminal string for $A$

 \[
S\implies a^3Sc^2 \implies  a^6Xc^7 \implies a^6bXc^7\implies a^6b^2c^7
.\] 

\section*{Sort Hand Notation}
an easier way of writing the grammar for $A$ in shorthand notation is as follows:
\begin{align*}
        S&\to a^3Sc^2 \mid a^3Xc^5 \\
        X&\to bX  \mid b
.\end{align*}
\end{document}
