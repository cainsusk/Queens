\documentclass[12pt]{book} 

\usepackage{amsmath}
\usepackage{graphicx}
\usepackage{import}
\usepackage{booktabs}

\setlength{\parindent}{0em}  % sets auto indent at new paragraph to none

\newcommand{\incfig}[1]{%
    \import{./figures/}{#1.pdf_tex}
}

\title{\coursetitle\linebreak\lecturename}
\author{\\Cain Susko\\ 
           \\ \\ \\
      Queen's University 
    \\School of Computing\\} 

%=-=-=-=-=-title-=-=-=-=-=%
\newcommand{\lecturename}{Deterministic State Diagram Minimization Algorithm}
\newcommand{\coursetitle}{Software Specifications}
%=-=-=-=-=-#####-=-=-=-=-=%

\begin{document}
\begin{titlepage}
        \maketitle
\end{titlepage}


\section*{Minimization Algorithm}
The algorithm for minimizing a state diagram is the following (note: must do preprocessing see lesson 22)
\begin{enumerate}
        \item mark a pair of states $(A,B)$ where  $A$ is final and  $B$ is non-final or vice versa. 
                (distinguished by the empty-string $\epsilon$)
        \begin{enumerate}         
                \item consider pairs $(A,B)$ that remain unmarked. 
                        If on a single alphabet symbol  $b\in\Sigma$ makes the states  $(A,B)$ reach a pair  $(C,D)$ and $(C,D)$
                        was already marked, then mark pair $(A,B)$.
                \item repeat from step 1 until there are no more pairs $(A,B)$ that are a final and end state and no more 
                        pairs $(C,D)$ (see part a).
        \end{enumerate}

        \item all unmarked state-pairs are indistinguishable and each pair is merged.
\end{enumerate}

This algorithm has a complexity of $O(n^3)$ where  $n$ is the number of states.

Note: If a pair  $(p,q)$ is distinguishable, so is  $(q,p)$ thus, the algorithm need only consider half of the $n^2$ pairs.

\section*{Example}
consider the given state diagram:
\begin{figure}[h]
        \centering
        \incfig{ex1}
\end{figure}

Where the nodes are:\\ top-left = $w$, top-right = $x$, bottom-left = $y$, bottom-right = $z$
We can then make a matrix representing the pairs, for this problem. 
Note that because we only need to consider half the pairs, we only need to fill in half the matrix\\

\subsection*{Stage 0}
mark pairs where one state is final and the other is not with 0
(note that the diagonal is filled with X as a state is indistinguishable from its self)

\begin{table}[ht]
\centering
\begin{tabular}{@{}lllll@{}}
\toprule
  & w & x & y & z \\ \midrule
w & X &   &   &   \\
x & 0 & X &   &   \\
y & 0 &   & X &   \\
z & 0 &   &   & X \\ \bottomrule
\end{tabular}
\end{table}

\subsection*{Stage 1}
\begin{itemize}
        \item $(x,z)\rightarrow^b (w,z)$, $(w,z)$ has been marked, thus mark  $(x,z)$
        \item $(x,y)\rightarrow^a(z,z)\;\;\wedge\rightarrow^b(w,w)$,  $(z,z)\wedge(w,w)$ has not been marked, thus $(x,y)$  is not marked,
        \item $(y,z)\rightarrow^b(w,z)$, $(w,z)$ is marked, thus mark  $(y,z)$
\end{itemize}
thus, we have marked all states that can be marked as per algorithm step 1.b.
\begin{table}[ht]
\centering
\begin{tabular}{@{}lllll@{}}
\toprule
  & w & x & y & z \\ \midrule
w & X &   &   &   \\
x & 0 & X &   &   \\
y & 0 &   & X &   \\
z & 0 & 1 & 1 & X \\ \bottomrule
\end{tabular}
\end{table}
\pagebreak


\subsection*{stage 2}
Now we have $(x,y)$ as our only resulting indistinguishable pair.
Thus, we will merge  $x,y$ to get our final unique minimized Deterministic State Diagram.
 \begin{figure}[h]
         \centering
         \incfig{ex2}
        
\end{figure}

Note: there is only one minimal state for each language
\end{document}

