\documentclass[12pt]{book} 

\usepackage{amsmath}
\usepackage{graphicx}
\usepackage{import}

\setlength{\parindent}{0em}  % sets auto indent at new paragraph to none

\newcommand{\incfig}[1]{%
    \import{./figures/}{#1.pdf_tex}
}

\title{\coursetitle\linebreak\lecturename}
\author{\\Cain Susko\\ 
           \\ \\ \\
      Queen's University 
    \\School of Computing\\} 

%=-=-=-=-=-title-=-=-=-=-=%
\newcommand{\lecturename}{Closure Properties of Regular Languages}
\newcommand{\coursetitle}{Software Specifications}
%=-=-=-=-=-#####-=-=-=-=-=%

\begin{document}
\begin{titlepage}
        \maketitle
\end{titlepage}


\section*{Closure Properties}
Regular languages are closed under Boolean operations thus, if $R$ and $S$ are regular, then so are:
 \begin{itemize}
        \item $R\cup S$ (=$R+S$)
        \item  $R\cap S$
        \item  $\overline{R} = \Sigma\text{*} -R$ (complement of $R$)
\end{itemize}

To illustrate this property we will show the state diagram fir language $L$:
 \begin{figure}[h]
        \centering
        \incfig{L1}
\end{figure}

And then the closure of diagram $L$,
\begin{figure}[h]
        \centering
        \incfig{L2}
\end{figure}

Such that every time $L$ is in a final state, its closure is not.
Note that this can only be done with Deterministic State Diagrams.
\pagebreak

\subsection*{Example}
Given:
\[A = \{w\in\Sigma  \mid  w\text{ has equally many many occurrences of symbols a and b} \}\]

Show that $A$ is non-regular\\
Recall:
\[B = \{a^i b^i \mid i \geq 0\}\]
Note:
\[A\cap a\text{*}\;b\text{*} = B\]
If A were regular, then is B (because the union of 2 regular languages is also regular). We know B is not regular thus: A is not reguar
\end{document}

