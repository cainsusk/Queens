\documentclass[12pt]{book} 

\usepackage{amsmath}
\usepackage{graphicx}
\usepackage{import}

\setlength{\parindent}{0em}  % sets auto indent at new paragraph to none

\newcommand{\incfig}[1]{%
    \import{./figures/}{#1.pdf_tex}
}

\title{\coursetitle\linebreak\lecturename}
\author{\\Cain Susko\\ 
           \\ \\ \\
      Queen's University 
    \\School of Computing\\} 

%=-=-=-=-=-title-=-=-=-=-=%
\newcommand{\lecturename}{Context-Free Grammar Example}
\newcommand{\coursetitle}{Software Specifications}
%=-=-=-=-=-#####-=-=-=-=-=%

\begin{document}
\begin{titlepage}
        \maketitle
\end{titlepage}


\section*{Example}
Given the following, find the grammar for $B$:
 \begin{align*}
         \Sigma &= \{a,b,c,d\}\\
         B &= \{a^ib^{2k}c^kd^{3i} \mid i,k \geq 1\}\cup\{a^rb^{2r}c^sd^{3s} \mid r,s\geq 1\}
.\end{align*}

We can see that in both sets, $a,b$ and  $c,d$ are related in the first set and $a,b$ and $c,d$ re related in the second set
        because of the exponents.
Note that union is like saying \textit{or}. Thus the first rule is
\[S\to X \mid Y\]
now we can continue with defining $X$ and  $Y$.

We can then define $X$ as an option between it's two related parts:

\[S\to aXd^3 \mid aZd^3\]

Note that $X$ is in the first option because it can continue to make $a,d$ until it switches to the  $Z$ option where it will make
        the inner part of the string.
\[
Z\to b^2Zc \mid b^2
.\] 
Where the second option is the terminal string for $X$

Now we can move on to  $Y$
 \[
Y\to UW
.\] 

Note that capital letters like $W,U$ are variables. 
The above rule represents the concatenation of the two strings  $U,W$\\
Now, to define U\@:
 \[
U\to aUb^2  \mid ab^2
.\] 

which represents $U$'s recursive and terminal strings.\\
Finally, W is written as
 \[
W\to cWd^3  \mid cd^3
.\] 

Which has a similar form to $U$
\\

Thus we have found the recursive grammar rules for the union of sets $B$


\end{document}

