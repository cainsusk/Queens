\documentclass[12pt]{book} 

\usepackage{amsmath}
\usepackage{graphicx}
\usepackage{import}
\usepackage{amsfonts}
\usepackage{booktabs}

\setlength{\parindent}{0em}  % sets auto indent at new paragraph to none

\newcommand{\incfig}[1]{%
        \import{./figures/}{#1.pdf_tex}
}

\newcommand{\incimg}[2]{%
       \begin{figure}[h]
               \centering
               \includegraphics[scale = #2]{./figures/#1}
       \end{figure}
}

\title{\coursetitle\linebreak\lecturename}
\author{\\Cain Susko\\ 
           \\ \\ \\
      Queen's University 
    \\School of Computing\\} 

%=-=-=-=-=-title-=-=-=-=-=%
\newcommand{\lecturename}{Intro to Computability}
\newcommand{\coursetitle}{Software Specifications}
%=-=-=-=-=-#####-=-=-=-=-=%

\begin{document}
\begin{titlepage}
        \maketitle
\end{titlepage}


\section*{Computability}
Computability is an algorithmic problem with:
\begin{itemize}
        \item infinite set of inputs
        \item function that associates an output to input.
\end{itemize}

The solution to this problem is an algorithm that computes the function. For 
example: for each input the function computes the correct output.

\paragraph{Un-Computability}
may well defined algorithms problems are unsolvable, which is to say, an 
algorithm \textbf{does not exist}.

\section*{Un-Computable Example}
The specifications for the halting problem are as follows:
given 2 parameters \texttt{func} and \texttt{arg}:
\begin{itemize}
        \item[i] function returns true if the file \texttt{func} contains a
                $C$ function definition with one file parameter, and the function
                terminates if applied to the file \texttt{arg}
        \item[ii] function returns false otherwise
\end{itemize}

which means that:
the function \textbf{returns true} if \texttt{func} contains a function 
which halts given the input \texttt{arg}

\paragraph*{Proving Un-Computability}
The Technique used for proving that a function is un-computable is called
digitalization. For the sake of contradiction, assume we can define:
\[\texttt{halts(FILE *func, FILE *arg) \{...\}}\]
\pagebreak

The implementation of the \texttt{halts} function is as follows:
\begin{verbatim}
int test(FILE *f){
        FILE *a = tmpfile();
        copy(f, a);
        if (halts(f, a)) for (j j) {...}
        return 0;
}
int main(void){
        FILE *f = tmpfile();
        fprintf(f, ...)
        /* write to file f the definition
        of test() */
        return test(f);
}
\end{verbatim}

In the main program, the \texttt{test()} function is called with 
a parameter that contains the definition of \texttt{test()} such that:
\begin{enumerate}
        \item[i] if inside \texttt{test()}, halts returns true then
                \texttt{test()} with it's own description which goes into
                an infinite loop, which causes \texttt{halt} to
                return an incorrect answer.
        \item[ii] if halt returns false then \texttt{test()} on its 
                own description terminates normally, again \texttt{halt}
                has returned an incorrect answer.
\end{enumerate}

Thus, in all cases \texttt{halts} returns an incorrect answer and is a 
contradiction. Therefore, \texttt{halts} cannot be implemented.

\section*{Dependence on Language}
From a computability point of view, all general purpose languages
are equivalent; which is to say they can implement the same
class of functions. This has been proven by using the fact that
many languages compile to the same low-level language. Furthermore,
there exists simple theoretical models that are equivalent (in terms
of computability) to general purpose programming languages. For example: 
Turing Machines.

Strictly speaking, we have only proved that \texttt{halts} cannot be implemented
in \textit{C}. However, the argument used above will owrk for any programming 
language (for \texttt{halts}).

\end{document}

