\documentclass[12pt]{book} 

\usepackage{amsmath}
\usepackage{graphicx}
\usepackage{import}
\usepackage{amsfonts}
\usepackage{booktabs}

\setlength{\parindent}{0em}  % sets auto indent at new paragraph to none

\newcommand{\incfig}[1]{%
        \import{./figures/}{#1.pdf_tex}
}

\newcommand{\incimg}[2]{%
       \begin{figure}[h]
               \centering
               \includegraphics[scale = #2]{./figures/#1}
       \end{figure}
}

\title{\coursetitle\linebreak\lecturename}
\author{\\Cain Susko\\ 
           \\ \\ \\
      Queen's University 
    \\School of Computing\\} 

%=-=-=-=-=-title-=-=-=-=-=%
\newcommand{\lecturename}{Computable Specifications - Reducibility}
\newcommand{\coursetitle}{Software Specifications}
%=-=-=-=-=-#####-=-=-=-=-=%

\begin{document}
\begin{titlepage}
        \maketitle
\end{titlepage}


\section*{The Church-Turing Thesis}
The thesis claims that any algorithmic problem that has an informed algorithm can be solved
by a Turing machine. This thesis is \textit{generally} believed to be true, however it is not
possible to prove the CR thesis as `informed algorithm' does not have a precise definition.

\section*{Reducibility}
The process described in the previous note was long and tedious, and not something we want to
do every time we want to show something is incomputable. Thus, we can use \textbf{reduction}
in order to show that a function is incomputable by taking a function that has already been 
proven to be incomputable, and reduce the proven function to the unproven one.

\paragraph{Algorithm}
function $A$ reduces to function $B$ \textbf{if} an algorithm for $B$ can be used to construct
an algorithm for $A$

When $A$ is reducible to $B$, it can be possible that:
\begin{itemize}
        \item $A$ is computable then $B$ is incomputable
        \item $A$ and $B$ are both computable
        \item both $A$ and $B$ are incomputable 
\end{itemize}

\subsection*{Example}
given:
\[\texttt{bool haltsOnEmpty(FILE *func)}\]
which returns \textbf{true} if func contains a definition of an
int function with one file parameter and that function halts when 
applied to the empty file. Returns \textbf{false} otherwise.

From an initial observation, one may be able to note that \texttt{haltsOnEmpty} reduces to 
\texttt{halts} as it is a special case of \texttt{halts}. Thus, in order
to show \texttt{haltsOnEmpty} is incomputable, we must reduce it to \texttt{halts}.

Note: The \texttt{halts} problem is the only known incomputable function.
\pagebreak

In order to reduce the function, we must implement a file transformation such that:
\[\texttt{merge(func, arg, funcArg}\]

where if \texttt{func} contains a function \texttt{int f(FILE *a)\{C\}}, then
merge writes to FUncArg following definition:
\begin{verbatim}
inf f(FILE *b)
{ 
        FILE *a = tmpfile();
        
        fprintf(a, "%s", "arg");

        C # code C refers to file a
}
\end{verbatim}

Now that we have the reduced code, we must show you can create \texttt{halts} with
\texttt{haltsOnEmpty}:
\begin{verbatim}
bool halts (FILE *func, FILE *arg){
        FILE *funcArg = tmpfile();
        merge(func, arg, funcArg);

        return haltsOnEmpty(funcArg);
\end{verbatim}

This works because the function written to \texttt{funcArg} operates
like \texttt{func} on input \texttt{arg}.

Assuming that \texttt{haltsOnEmpty} satisfies it's specification then the
above program implements \texttt{halts}

Therefore, \texttt{haltsOnEmpty} is incomputable.

\end{document}

