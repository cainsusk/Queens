\documentclass[12pt]{book} 

\usepackage{amsmath}
\usepackage{graphicx}
\usepackage{import}
\usepackage{amsfonts}
\usepackage{booktabs}

\setlength{\parindent}{0em}  % sets auto indent at new paragraph to none

\newcommand{\incfig}[1]{%
        \import{./figures/}{#1.pdf_tex}
}

\newcommand{\incimg}[2]{%
       \begin{figure}[h]
               \centering
               \includegraphics[scale = #2]{./figures/#1}
       \end{figure}
}

\title{\coursetitle\linebreak\lecturename}
\author{\\Cain Susko\\ 
           \\ \\ \\
      Queen's University 
    \\School of Computing\\} 

%=-=-=-=-=-title-=-=-=-=-=%
\newcommand{\lecturename}{Verification of Conditional Statements in Specifications}
\newcommand{\coursetitle}{Software Specifications}
%=-=-=-=-=-#####-=-=-=-=-=%

\begin{document}
\begin{titlepage}
        \maketitle
\end{titlepage}


\section*{If Else Verification}
there are 2 parts to verification, given the example:
\begin{verbatim}
       ASSERT(y == y0 && z == z0)

       if(y <= z)   {y = z+1; z = z+2;} else
                    {z = y+2; y = y+1;}

       ASSERT(y == y0 && z == z0 && y <= z)
\end{verbatim}
\begin{enumerate}
        \item[\textbf{If Part}] FINISH WITH SLIDES ON ONQ
\end{enumerate}

\section*{If Verification}
there is also a case of the if statement when there is no else part. this is represented as:
\[P\;if\;(B)\;C\}\;Q\]

the inference rule for this statement is:
\[\frac{P\;\&\&\;B\;\{C\}\;Q\;\;\;\;\;\;\;\;\;P\;\&\&!\;B\{\;\}\;Q}{P\;\{if\;(B)\;C\}\;Q}\]

\paragraph{Example}
\begin{verbatim}
        ASSERT(z <= y)
        if (w > z || y > x) {w = z-1; x = y;}
        ASSERT(w <= z <= y <= x)
\end{verbatim}

The If Part of the Verification looks like so:
\begin{verbatim}
       ASSERT(z <= y && (w > z || y > x)) 
       // z <= implies 
        FINISH USING SLIDE ON ONQ
\end{verbatim}

The non-existant else part would go like so:
\begin{verbatim}
        z <= y && !(w > z || y > x)
        z <= y && w <= z && y <= x
\end{verbatim}
in which the last line implies the post-condition in the if part.

\section*{Verifying While Loops}
To verify while loops a special assertion must be used. it is called the \textit{loop-invariant}. The loop-invariant has the 
following formula:
\[\frac{I\;\&\&\;B\;\{C\}\;I}{I\;\{while(b)\;C\;\}\;I\;\&\&\;!B}\]

verifying whie loops has the following proof tableau scheme:
\begin{verbatim}
        ASSERT(I)
                while(B){
                        ASSERT(I && B)
                                C
                        ASSERT(I)
                        } // end while
                ASSERT(I && !B)
\end{verbatim}

Observe: the inner assertion block (I \&\& B) -> (I) is the permise for the proof.
This proof also has a side condition that the evaluation of B should not change the state

\paragraph{What Should we use as Invariant}
there is a certian art to choosing which part of the code fragment to use as invariant, which must be done in order to verify the while loop

\paragraph{Example}
The best way to get gud is through doing so here is an example:
\begin{verbatim}
        ASSERT(true)
        i = 0; j = 100;
        while(i <= 100){

                i = i + 1;
                j = j - 1;} // end while
        ASSERT(i == 101 && j == -1)
\end{verbatim}

Here, the invariant is \texttt{i + j == 100} as this statement is always true within the loop.
the syntax for writing this within our above code would be:
\[\texttt{INVAR(i + j == 100 \&\& 0 <= i <= 101)}\]
Note: \texttt{0 <= i <= 101} is also invariant as it is always true within the loop
\end{document}

