\documentclass[12pt]{book} 

\usepackage{amsmath}
\usepackage{graphicx}
\usepackage{import}
\usepackage{amsfonts}
\usepackage{booktabs}

\setlength{\parindent}{0em}  % sets auto indent at new paragraph to none

\newcommand{\incfig}[1]{%
        \import{./figures/}{#1.pdf_tex}
}

\newcommand{\incimg}[2]{%
       \begin{figure}[h]
               \centering
               \includegraphics[scale = #2]{./figures/#1}
       \end{figure}
}

\title{\coursetitle\linebreak\lecturename}
\author{\\Cain Susko\\ 
           \\ \\ \\
      Queen's University 
    \\School of Computing\\} 

%=-=-=-=-=-title-=-=-=-=-=%
\newcommand{\lecturename}{Specifications}
\newcommand{\coursetitle}{Software Specifications}
%=-=-=-=-=-#####-=-=-=-=-=%

\begin{document}
\begin{titlepage}
        \maketitle
\end{titlepage}


\section*{Verifying Specifications}
given that the entries \texttt{A[0]...A[max-1]} are known to exist:
\begin{verbatim}
        ASSERT( 0 <= n <= max && A == A0 ) /* pre-condition */

        {int i
        A[n] = x; i=0;
        while (S[i] !+ x) i++;
        present = (i<n)

        ASSERT( (persent iff x in A[0:n-1]) && 
        ForAll (i=0; i<n) A[i] == A0[i] ) /* post-condition */
\end{verbatim}

So now, we want to create a systematic way of verifying if specifications hold.

for equalities, we can verify them by using the logic:
\[V == I :\{V = E_j\}\equiv V==[E](V\mapsto I)\]

but this forward logic is very convoluted. A better way may be to reason \textit{backwards}. Thus, based on the post condition, determine the
most general pre-contition that guarantees that the post-condition holds after assignment:
\[[Q](V \mapsto E) \{V = E_j\}\;Q\]
This is known as the Hoirre Axiom for finding pre-conditons.
assertion obtained by $Q$ by relacing occurences of $V$ with $E$.


Note: we must be careful when substituting quantified variables (will cover later).

\section*{Examples}
\begin{itemize}
        \item $P\;\;\{x = 1_j\}\;\;x==1$\\
                $P\;=\;1==1$ pre-condition is true
        \item $P\;\;\{x = 1_j\}\;\;x==0$\\
                $P\;=\;1==0$ pre-condition is false
        \item $P\;\;\{x = y + z_j\}\;\;x*x>y$\\
                $P\;=\;(y+z)*(y+z)>y$ pre-condition is unknown if true or false
\end{itemize}

\section*{Issues With Substitution}
there are a few special actions one should take in order to avoid problems with substitution using the Hoirre Axiom:
\begin{itemize}
        \item[i] Add parenthesis when neccesary
        \item[ii] Only free occurrences of a variable are substituted
        \item[iii] As a result of a substitution, free occurrences of a variable should not become bound. If neccessary, we should change the name of 
                the bound variable
\end{itemize}
\paragraph{Examples}
\begin{itemize}
        \item[i] $P\;\;\{z = x + y_j\}\;\; z*z > y$\\
                $P\;=\;(x+y)*(x+y) > y$
        \item[ii] $P\;\;\{z = x+y_j\}\;\;Exists(z=0, z<50)\;\;z==x+y$\\
                $P\;=\;Exists(z=0, z<50)\;\; z == x+y$
        \item[iii] $P\;\;\{z = x+y_j\}\;\;Exists(y=0, y < 50)\;\;x*y == z$ Note: we rename the bound variable $y_j$ to $a$\\
                $P\;=\;Exists(a=0, a<50)\;\;x*a == x+y$
\end{itemize}
note: we cannot rename free variables but we can rename bound variables.
\section*{Recall}
please Recall bound and unbound variables (from cisc204). 

given:
\[\exists(x = 0, x<w)\exists(z=a, z<b)\;\;[x*y \geq 2*z + w]\]
where the variables are:
\begin{itemize}
        \item[Free] $x, z$
        \item[Bound] $w, a, b, y$
\end{itemize}


\end{document}

