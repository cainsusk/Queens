\documentclass[12pt]{book} 

\usepackage{amsmath}
\usepackage{graphicx}
\usepackage{import}
\usepackage{amsfonts}

\setlength{\parindent}{0em}  % sets auto indent at new paragraph to none

\newcommand{\incfig}[1]{%
    \import{./figures/}{#1.pdf_tex}
}

\title{\coursetitle\linebreak\lecturename}
\author{\\Cain Susko\\ 
           \\ \\ \\
      Queen's University 
    \\School of Computing\\} 

%=-=-=-=-=-title-=-=-=-=-=%
\newcommand{\lecturename}{Ambiguity}
\newcommand{\coursetitle}{Software Specifications}
%=-=-=-=-=-#####-=-=-=-=-=%

\begin{document}
\begin{titlepage}
        \maketitle
\end{titlepage}


\section*{Ambiguity}
given the example:
\[
\diamond \to \diamond + \diamond  \mid  \diamond \times \diamond  \mid (\diamond ) \mid a
.\] 
Where this grammar's terminal string is:
\[
a+a\times a
.\] 
Upon examination of the string one may find that the string as 2 valid parse trees:
\begin{figure}[h]
        \centering
        \incfig{ambig1}
        \incfig{ambig2}
        \caption{the two parse trees of $a+a\times a$}
\end{figure}

Thus this means that this terminal string is \textbf{ambiguous}.
However, the tree on the right is the \textit{desired} parse tree as multiplication has higher precedence than addition 
        (we read the trees from the bottom up).
\paragraph{}
We can then use an `ad-hoc' process to transform this grammar example into an equivalent \textit{unambiguous} grammar. This 
        new grammar is as follows. Note that $\diamond = expression$
\begin{align*}
        expression &\to expressiom + term  \mid term\\
        term &\to term \times factor  \mid factor\\
        factor &\to (expression)  \mid a
.\end{align*}
\pagebreak


Thus the terminal string $a+a\times a$ in this grammar is unambiguous:
\begin{figure}[h]
        \centering
        \incfig{unambig}
\end{figure}

Thus this terminal string has a unique parse tree is unambiguous. To Prove a Grammar is unambiguous one must first show that 
        all terminal strings within the grammar are  unambiguous.






\end{document}

