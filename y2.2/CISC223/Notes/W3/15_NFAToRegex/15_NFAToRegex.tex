\documentclass[12pt]{book} 

\usepackage{amsmath}
\usepackage{graphicx}
\usepackage{import}

\newcommand{\classID}{Convert State Diagram to Regex}
\newcommand{\coursename}{Software Specifications}


\newcommand{\incfig}[1]{%
    \import{./figures/}{#1.pdf_tex}
}

\begin{document}
\date{}
\setlength{\parindent}{0em}  % sets auto indent at new paragraph to none

\title{\coursename\\\classID}

\author{\\ \\ Cain Susko\\\today \\ \\ \\ \\ \\
        Queen's University \\School of Computing} 
 

\maketitle

\pagebreak

\section*{State Elimination Algorithm}
This algorithm is used to covert a State Diagram to it's corresponding Regex.
The conditions for the algorithm are as follows:
\begin{enumerate}
        \item The State Diagram used as input should have exactly one final state which is not the start state
        \item If these conditions are not met, we modify the state diagram by adding a new final state with $\epsilon$--transitions 
                from the original final states
\end{enumerate}

\begin{figure}[h]
        \centering
        \incfig{algoex}
        \incfig{algoex2}
\end{figure}

\paragraph{}
The Algorithm uses an intermediate stage \textit{Generalized State Diagrams} where transitions are labelled by Regex.
The algorithm eliminates states one at a time.
The start state and final state are \textit{not} eliminated.\\
\pagebreak

Thus the elimination step is as such, where $U, V, T, R$ are Regex.\\
We eliminate the top node, the transition between the bottom 2 is now the combined Regex for the removed transitions.
\begin{figure}[ht]
        \centering
        \incfig{algoex3}
\end{figure}
\begin{figure}[ht]
        \centering
        \incfig{algoex4}
\end{figure}

Note, when eliminating a state we need to add transitions to \textbf{all pairs} of states connected to the eliminated state.
In particular, it is possible that 2 states are the same state.\\
\pagebreak

At the end, the algorithm should produce a 2-state diagram like so:
\begin{figure}[h]
        \centering
        \incfig{final1}
\end{figure}

Which can be translated into it's equivalent Regex.
\[R\text{*}U(S+VR\text{*}U)\text{*}\]

And thus, this is the state elimination algorithm for converting State Diagrams to their corresponding Regex
\end{document}

