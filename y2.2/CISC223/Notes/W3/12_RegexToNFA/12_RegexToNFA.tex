\documentclass[12pt]{book} 

\usepackage{amsmath}
\usepackage{graphicx}
\usepackage{import}

\graphicspath{ {./figures} }

\newcommand{\classID}{Convert Regex to State Diagram}
\newcommand{\coursename}{Software Specifications}

\newcommand{\incfig}[1]{%
    \import{./figures/}{#1.pdf_tex}
}

\begin{document}

\date{}
\setlength{\parindent}{0em}  % sets auto indent at new paragrahp to none

\title{\coursename\\\classID}

\author{\\ \\ Cain Susko\\\today \\ \\ \\ \\ \\
        Queen's University \\School of Computing} 
 

\maketitle
\pagebreak

\section*{Regex to $\epsilon$--NFA}
to do this we need 2 algorithms different algorithms
\begin{itemize}
        \item convert Regex to a State Diagram 
        \item convert State Diagram to Regex
\end{itemize}

The first algorithm will produce a NFA with $\epsilon$-transitions as well as the following properties:\\

\begin{enumerate}
        \item there is exactly one final state and it is not the start state
        \item the start state has no incoming transitions
        \item the final state has no outgoing transitions
\end{enumerate}
these conditions are needed to guarantee the correctness of the recursive steps of the algorithm\\
said algorithm is as follows:

The base cases for the algorithm (as its recursive) are $a\in\sum$,  $empty\;string$, and $\emptyset$

\begin{figure}[h]
        \centering
        \incfig{basecase}
\end{figure}
\pagebreak

Thus, the inductive step is:\\
suppose that a state diagram $S_i$ has been constructed for regex  $R_i$ where  $i=1,2$

\begin{figure}[ht]
        \centering
        \incfig{inductivestep}
        \centering
\end{figure}

Now, for $R_1 + R_2$:
\begin{figure}[h]
        \centering
        \incfig{rplusr}
\end{figure}

We must merge the start and end states.\\
note that conditions 1 and 2 (see list above) guarantee that any computation uses only transitions from $S_1$
        or only transitions from $S_2$
\pagebreak

For $R_1\cdot R_2$:

\begin{figure}[ht]
        \centering
        \incfig{rtimesr}
\end{figure}

We must merge the final state of $S_1$ with the starting state of $S_2$. 
This merged state is not final.
Conditions 2 and 3 guarantee that any accepting path consists of a string by $S_1$ concatenated with 
        a string accepted by $S_2$.\\

For $R_i*$   

\begin{figure}[h]
        \centering
        \incfig{rclosed}
\end{figure}

We must merge the start and end states for $R_i$\\
        furthermore the merged state is \textbf{not final}, and it cannot be the start state.
We must then add a start and end state (with $\epsilon$-transition) to complete the diagram.\\
we use this algorithm recursively, starting from the smallest part of a Regex, like a single number $R_i$, and build our way to
        a more complex diagram.


\end{document}
