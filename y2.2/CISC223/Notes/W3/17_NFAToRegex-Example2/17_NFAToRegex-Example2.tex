\documentclass[12pt]{book} 

\usepackage{amsmath}
\usepackage{graphicx}
\usepackage{import}

\newcommand{\classID}{Converting State Diagram to Regex-Examples cont.}
\newcommand{\coursename}{Software Specifications}


\newcommand{\incfig}[1]{%
    \import{./figures/}{#1.pdf_tex}
}

\begin{document}
\date{}
\setlength{\parindent}{0em}  % sets auto indent at new paragraph to none

\title{\coursename\\\classID}

\author{\\ \\ Cain Susko\\\today \\ \\ \\ \\ \\
        Queen's University \\School of Computing} 
 

\maketitle
\pagebreak

\section*{Example}

Given the following state diagram, convert it to its corresponding Regex using the State Elimination Algorithm from lesson 15.
The diagram only has 1 end state so there is no preprocessing necessary.
\begin{figure}[h]
        \centering
        \incfig{st1}
\end{figure}

After removing the Top Right node the final generalized graph for the given Regex is:
\begin{figure}[h]
        \centering
        \incfig{st2}
\end{figure}

And thus the resulting Regex is:

\[(ac\text{*}b)\text{*}(b+ac\text{*}a)[b+dc\text{*}a+(c+dc\text{*}b)(ac\text{*}b)(b+ac\text{*}a)]\text{*}\]

This State Diagram has a unique Regex because there is only one state that can be removed, as the other 2 are the 
        start and end states--which cannot be removed
\end{document}

