\documentclass[12pt]{book} 

\usepackage{amsmath}
\usepackage{graphicx}
\usepackage{import}

\setlength{\parindent}{0em}  % sets auto indent at new paragraph to none


\newcommand{\incfig}[1]{%
    \import{./figures/}{#1.pdf_tex}
}

\title{\coursetitle\linebreak\lecturename}

\author{\\Cain Susko\\ 
           \\ \\ \\
      Queen's University 
    \\School of Computing\\} 

%=-=-=-=-=-title-=-=-=-=-=%
\newcommand{\lecturename}{The Pumping Property of State Diagrams}
\newcommand{\coursetitle}{Software Specifications}
%=-=-=-=-=-#####-=-=-=-=-=%

\begin{document}
\begin{titlepage}
        \maketitle
\end{titlepage}


\section*{Regular Languages}
A regular language is a language that can be specified by a Regex. This type of language is well suited to be 
        converted to a State Diagram using the algorithm to convert a Regex to $\epsilon$-DFA from lesson 14. 
Furthermore, There are other algorithms that we have learned about that can convert a $\epsilon$-DFA to a DFA to a NFA\\
There are however, limitations to what a regular language can be.\\
For example:
\[\{a^ib^i | i \geq 0\}\]

The set of balanced strings (above) is not regular. 
This is because one would need an infinite amount of states to represent this Regex in a State Diagram 
        as each length of word would be a different state. 
Furthermore, the state diagram would have to accept illegal strings in order to represent the balanced set.

\section*{The Pumping Property of Deterministic State Diagrams}
Accepting computation on a string $x$ where $x$ has a length greater than the number of states
\begin{figure}[h]
        \centering
        \incfig{pp1}
\end{figure}

This shows the state diagram for a computation on $x_i$ such that:
\[x=p\cdot q^k\cdot r\]
where $x$ is accepted  $\forall_k\{k\geq 0\}$.
{\small Thus, this yields to us the nonregularity conditions formalized in the \textbf{pumping lemma}}

\end{document}

